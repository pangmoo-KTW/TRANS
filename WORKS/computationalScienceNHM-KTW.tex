\documentclass[a4paper,atbegshi,openany]{memoir}
\usepackage[dbl4x6]{fapapersize}
\usepackage{amsmath,amssymb,amsfonts,amsthm}
\usepackage{braket,hyperref,nicematrix}
\usepackage{enumitem,mdframed,}
\hypersetup{
  colorlinks=true,linkcolor=teal,filecolor=magenta,urlcolor=cyan,
  citecolor=magenta,
}
\chapterstyle{chappell}
\setlist{nosep}
\title{Computational Science II}
\author{Roh Hyeonmin \& Kim Taewon}
\date{\today}
\begin{document}
\maketitle\thispagestyle{empty}
\newpage
\tableofcontents
\chapter{Introduction}
\emph{Dequantization}, the process of presenting classical counterparts to 
specific quantum machine learning (QML) algorithms while incurring only a
polynomial slowdown, raises questions about the claimed exponential quantum
advantage of QML algorithms. At the core of QML and its asserted advantages
lies HHL algorithm, developed by Harrow, Hassidim and Lloyd \cite{HHL2009}.
It's noteworthy that Aaronson \cite{Aaronson2015} critiqued the HHL algorithm for
its intricate `fine print' conditions, which also influence QML algorithms
rooted in the HHL approach. 

A few years later, Tang \cite{Tang2019} dequantified the quantum recommendation
algorithm initially proposed by Kerenidis and Prakash \cite{KP2017} with just
a polynomial slowdown. This achievement was facilitated by the inherent
similarities between quantum techniques, such as `quantum phase estimation',
and classical linear algebra methods, exemplified by techniques like 
`$\ell^2$-norm sampling through singular decomposition,' introduced several
years ago \cite{Frieze2004}.

The central implication of this outcome is that if quantum linear algebra
algorithms can be efficiently dequantized, it prompts us to reevaluate the 
fundamental concept and applicability of the term ``quantum'' itself concerning
algorithms. However, as Cotler, Huang, and McClean \cite{Cotler2021} assert,
such skepticism loses its significance when dequantization is applied to data
originating from quantum systems. In cases where classical computation cannot
accurately capture experimental quantum data, QML undeniably offers large
speedups. Therefore, such doubt may be resolved through the accumulation of an
extensive amount of experimental quantum data, often referred as a matter of 
`Quantum Random Access Memory' (QRAM). This is the direction of above three
with Preskill and others \cite{Huang2022} take.

Nevertheless, the uncharted territory at the intersection of quantum linear 
algebra and classical linear algebra remains an area ripe for exploration. This
paper delves with clarity into the historical overview provided above and 
ultimately underscores its significance by addressing a recently suggested 
problem on `classical and quantum singular value transformation' by Bakshi and
Tang \cite{Bakshi2023}.
\chapter{HHL Algorithm}
\chapter{Recommendation Algorithm}
\chapter{Experimental Quantum Data}
\chapter{Quantum Linear Algebra}
\bibliography{ref}
\bibliographystyle{plain}
\end{document}
