\documentclass[a4paper,atbegshi,chapter]{oblivoir}
\usepackage[dbl4x6]{fapapersize}
\usepackage{amsmath,amssymb,amsfonts,amsthm}
\usepackage{graphicx,xcolor,caption}
\usepackage{braket,hyperref,nicematrix}
\usepackage{euler,enumitem,mdframed}
\usepackage{ob-chapstyles}
\chapterstyle{obchappell}
\setlist{nosep}
\hypersetup{
  colorlinks=true,linkcolor=teal,filecolor=magenta,urlcolor=cyan,
}
\newtheorem{defn}{정의}[chapter]
\newtheorem{theo}{정리}[chapter]
\newtheorem{thes}{논제}[chapter]
\newtheorem{axio}{공리}[chapter]

\title{\Large Quantum and quantum-inspired linear algebra (Tang, 2023) 학습일지}
\author{김태원}
\date{최초 작성 : 2023년 9월 4일 \\ 최근 편집 : \today}
\begin{document}
\maketitle
\chapter{블록-인코딩 개요}
양자계{\tiny quantum systems} 시뮬레이션은 양자컴퓨터를 만들려는 최초의 동기
가운데 하나였다. 이들 시뮬레이션 가운데 가장 간단한 것은 해밀토니언{\tiny 
Hamiltonian} 시뮬레이션이다.  해밀토니안 $H$는 계의 총 에너지에 대응하는 
연산자이며, 해밀토니언 시뮬레이션은 1982년 파인만{\tiny Richard
Feynman}이 처음 고안한 양자정보학 문제다. 해밀토니안 시뮬레이션이 계의 크기에
따라 지수적인{\tiny expnential} 증가를 보이는 탓에 양자컴퓨터가 필요할지도
모른다는 것이 파인만의 요지였다.

\begin{mdframed}
\textbf{해밀토니언 시뮬레이션.}\\ $H$가 파울리{\tiny Pauli} 행렬 $m$개로 구성한
    해밀토니언이라고 하자. 이는 다음을 뜻한다.
    \[
      E_a\textrm{가 파울리 행렬들의 텐서곱일 때, }H = \sum_{a=1}^m\lambda_a E_a.
    \]
    $e^{-iHt}$에 근접하여 $\|U-e^{iHt}\|\leq\epsilon$인 유니타리 $U$를
    구현하는 알고리즘을 찾아라.
\end{mdframed}

여기서 (1) 파울리 행렬, (2) 파울리 행렬들의 텐서곱, (3) $e^{-iHt}$, (4) 유니타리
같은 용어가 낯설 수 있겠다. 우선 파울리 행렬의 정의에 유니타리의 정의가 
필요하므로 유니타리 $U$부터 정의한다.
\begin{defn}
  복소 정방행렬 $U$는 켤레전치 행렬 $U^{\dagger}$가 $U$의 역행렬이기도 할 때 
  유니타리다. 즉 항등행렬 $I$에 대해 유니타리 행렬 $U$는 다음과 같다.
  \[
    U^{\dagger}U = UU^{\dagger} = UU^{-1} = I
  \]
\end{defn}
유니타리 행렬 혹은 변환은 어느 유클리드 노름 단위 벡터를 다른 유클리드 노름
단위 벡터로 사상한다. 여기서 `유클리드 노름 단위 벡터'를 큐비트{\tiny qubit}라고
부를 수도 있다. 

파울리 행렬은 유니타리 뿐만 아니라 에르미트{\tiny Hermitian},
거듭{\tiny involutory} 행렬로 정의되므로 에르미트와 거듭 행렬도 살핀다.
\begin{defn}
  에르미트 행렬 $A$는 정방복소행렬로, $A$의 켤레전치 $A^{\dagger}$와 $A$가
  같은 행렬이다.
\end{defn}
\begin{defn}
  거듭행렬은 $A$가 $A$ 자신의 역행렬인 정방행렬이다. 다시 말해 $A$가 거듭{\tiny
  involution}인 필요충분조건은 $A^2=I$다.
\end{defn}
\begin{defn}
  파울리 행렬은 $2\times2$ 복소행렬 세 개의 집합으로, 이들 복소행렬은 에르미트,
  유니타리, 거듭 행렬이다. 
  \[
    \sigma_1 = \sigma_x = \begin{pmatrix}0 & 1 \\ 1 & 0\end{pmatrix},\quad
    \sigma_2 = \sigma_y = \begin{pmatrix} 0 & -i \\ i & 0\end{pmatrix},\quad
    \sigma_3 = \sigma_z = \begin{pmatrix}1 & 0 \\ 0 & -1\end{pmatrix}
  \]
\end{defn}
텐서곱 $\otimes$은 아래처럼 정의된다.
\[
  x\otimes y \iff (x\otimes y)_{ij}=x_iy_j.
\]
그리고 어떤 시간 $t$에 상태 $\ket{\alpha}$가 주어진다고 하자. 시간이 $t=T$일 때
상태는 $\ket{\alpha(T)}$로 나타내는데 이는 $\ket{\alpha}$에 유니타리 (시간 변화) 
연산자 $U_1$를 적용한 것과 같다. 여기서 
\[
  U_1 = e^{-iHT}
\]
$H$는 해밀토니언을 나타내고, $T$는 시간을 나타낸다. 반면 해밀토니언
시뮬레이션에서는
\[
  \|U-e^{iHt}\|\leq\epsilon
\]
이므로 구해야 하는 유니타리 $U$는 해밀토니언 $H$와 시간 $T$에 대한 시간 변환
연산 $U_i$에 근사하는 연산이다. 그리고 해밀토니언 $H$는 상호작용하는 항들인
$E_a$의 합으로 볼 수 있고 이때 $\lambda_a$는 그런 상호작용의 강도를 결정한다.

보통 이 문제의 답은 충분히 큰 $r$에 대해 아래와 같은 트로터{\tiny Trotter}
근사로 주어진다.
\[
  e^{-iHt}\approx (e^{-iE_1t/r}e^{-iE_2t/r}\ldots e^{-iE_mt/r})^r.
\]
\end{document}
