\documentclass[a4paper,atbegshi,chapter,itemph,hidelinks]{oblivoir}
\usepackage[dbl4x6]{fapapersize}
\usepackage{amsmath,amssymb,amsfonts,amsthm}
\usepackage{braket,hyperref,nicematrix}
\usepackage{euler,enumitem,mdframed}
\setlist{nosep}
\title{계산과학2 (1)\\\textbf{주제: 역양자화}}
\author{노현민-김태원 조}
\date{\today}
\begin{document}
\maketitle
\newpage
\tableofcontents
\chapter{개요}
\textbf{역양자화\footnotesize dequantization}란 양자 기계학습{\footnotesize
quantum machine learning} 알고리즘을 분석하는 이론적인 도구로, 양자 판본보다 
오직 다항시간상으로만{\footnotesize polynomially} 느린 고전 (데이터에 대한) 
기계학습 알고리즘을 제시하여 양자 기계학습 알고리즘이 취한다고 알려진
지수적인 양자 우위{\footnotesize exponential quantum advantage}를 의문에
부치는 작업이다. 여기서 양자 기계학습 혹은 `QML'은 Harrow, Hassidim,
Lloyd의 \snm{선형방정식계에 대한 양자 알고리즘}{\footnotesize Quantum Algorithm
for Linear Systems of Equations, 2009}, 이른바 \emph{HHL 알고리즘}으로 발아한
분야다. 이에 Aaronson 같은 이론가는 \snm{세부조항을 읽으시오}{\tiny Read the
fine print, 2015}와 같은 비평을 남겼는데, HHL 알고리즘과 이를 바탕으로 삼은
온갖 QML 알고리즘들이 너무 많은 조건을 요구한다는 요지였다. 3년 지나, Tang은
HHL 알고리즘과 마찬가지로 `양자 위상 예측{\footnotesize quantum phase 
estimation}'을 사용하는 Kerenidis와 Prakash의 양자 추천 알고리즘{\tiny Quantum
recommendation systems, 2017}을 역양자화하여, 다시 말해 양자 위상 예측을 `특잇값
분해에 의한 $\ell^2$ 노름 샘플링'과 대치하여 이 양자 추천 알고리즘에 대한
고전 추천 알고리즘으로 내놓는다. \snm{추천 시스템에 대한 양자 알고리즘의 영향을
받은 고전 알고리즘}{\tiny A quantum-inspired classical algorithm for 
reccommendation systems, 2019}에 따르면 이 역양자화 알고리즘은 양자 판본에
대해 오직 다항시간상의 감속을 보일 뿐이었다. 역양자화는 일견 기예처럼
보이기도 하지만, 결정적인 함의는 대단히 기초적인 성격을 지닌다. 요컨대, 어떤 양자
선형대수 알고리즘을 고전 알고리즘으로 (다항시간상의) 모사하기 어렵지 않다면, 이들
양자 선형대수 알고리즘에서 `양자적인 것'이 무엇이냐고 반문할 수 있다. 한편
이처럼 `양자적인 것'을 탐구하고자 한다면 역양자화가 제시하는 한계를 적극
사용할 수 있을 테며, 조만간 양자컴퓨터를 기계학습이라는 분야에 도입하고자 한다면
이러한 의사결정에 역양자화의 결과가 경제적인 영향을 미칠 수 있다.

\chapter{HHL}
\end{document}
