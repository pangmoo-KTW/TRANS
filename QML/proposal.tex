\documentclass[a4paper,hidelinks]{oblivoir}
\usepackage{fapapersize}
\usefapapersize{*,*,30mm,*,30mm,*}
\begin{document}
\thispagestyle{empty}
\begin{center}
  \underline{계산과학 2 노현민-김태원 조 과제 제안서} \\
  \textbf{\large역양자화 알고리즘 및 양자 선형대수} \\ 
  \today\; 김태원 작성
\end{center}
\emph{역양자화\footnotesize dequantizing} 혹은 \emph{양자적 영향상의
고전 알고리즘\footnotesize quantum-inspired classical algorithm}은 추천
시스템이나 위상학적 데이터 분석 같은 양자 기계학습{\footnotesize QML}
알고리즘이 취한다고들 하는 고전 기계학습 알고리즘에 대한 계산적인 
우위{\footnotesize computational advantage}의 정체를 밝히는 분야다.

애론슨{\footnotesize Scott Aaronson}의 서베이 \snm{세부 조항을 
읽으시오}{\footnotesize(\emph{Read the fine print}, Nature Physics 2015)}에 따르면
HHL{\footnotesize Harrow-Hassidim-Lloyd} 알고리즘처럼 지수적인 가속{\footnotesize
exponential speedup}을 보인다고 주장하는 양자 기계학습 알고리즘들은 아주
작은 글자로 적힌 주의사항처럼 결정적인 가정 내지 조건을 은연중에 요구한다.
이들 가정은 양자 기계학습 알고리즘과 고전 기계학습 알고리즘 간의 정당한
비교를 어렵게 만든다.

2018년 애론슨은 학부연구생 탕{\footnotesize Ewin Tang}에게
양자 기계학습 추천 알고리즘\footnote{\emph{Quantum recommendation systems},
  Kerenidis \& Prakash, Innovations in Theoretical Computer Science, 
2017.}이 종래 기계학습 추천 알고리즘에 대해 취한다는 지수적인 가속의
계산복잡도이론적 증명을 연구 주제로 제시한다. 하지만 탕은 그 반증을
내놓는다. 양자 기계학습 추천 알고리즘에서 요구하는 \emph{양자 위상
예측\footnotesize quantum phase estimation}이라는 가정이 고전 기계학습
알고리즘에서 가정할 수 있는 \emph{특이값 분해{\footnotesize
singular value estimation} 꼴의 $\ell^2$ 노름 샘플링}\footnote{\emph{Fast
Monte-Carlo Algorithms for finding Low-Rank Approximations}, Frieze \& Kannan
\& Vempala, Journal of the ACM, 2004.}과 본질적으로 다르지 않다는 사실을 포착한
것이다. 그리하여 탕은 양자 기계학습 추천 알고리즘을 역양자화 다시 말해 그 영향을
받은 고전 기계학습 추천 알고리즘으로 내놓는다. 그리고 이 고전 기계학습 추천
알고리즘은 양자 판본에 대해 오직 다항시간상의{\tiny polynomial} 감속을 보일
뿐이었다.

탕은 위와 같은 내용을 학부 졸업논문 \snm{추천 시스템에 대한 양자적 영향상의
고전 알고리즘}{\footnotesize(A quantum-inspired classical algorithm for
reccommendation systems, STOC 2019)}으로 담았고 오늘날은 역양자화라는 핵심
기법을 여타 양자 기계학습 알고리즘에 대해 일반화하고자 한다. 여기서 양자 기계학습 
알고리즘이라는 말은 사실상 양자 선형대수 알고리즘을 뜻한다. 그러니 역양자화의
핵심 함의란 어떤 양자 선형대수 알고리즘은 고전 알고리즘에 의한 (다항시간 내의) 
모사가 어렵지 않으므로 이들 양자 선형대수 알고리즘에서 `양자'라는 표현의 본질이
무엇이냐는 물음이다. 

위와 같은 물음을 소화하려면 유니타리를 몇 개의 게이트로 구현하는 양자 회로인 
`블록 인코딩{\footnotesize block-encoding}', 블록 인코딩에 대한 자료구조, 이들
자료구조를 중첩{\tiny superposition}으로 질의{\tiny query}할 수 있는 양자 RAM
혹은 `QRAM'이라는 가정 등, 또 무엇보다 특잇값 분해에 대한 이해가 필요하다.
거칠게 말해 양자 기계학습 알고리즘의 성능은 양자 실험으로 생성한 훈련 데이터에
의존하며 이는 대개 QRAM 덕분에 중첩상의 입력으로 접근할 수 있기 
때문\footnote{\emph{Quantum advantage in learning from experiements}, Huang \&
Broughton \& Cotler \& Chen \& Li \& Mohseni \& Neven \& Babbush \& Kueng \&
Preskill \& Mcclean, Science, 2022.}이라고 한다.

본 과제를 통해 해결하고자 하는 문제는 양자 기계학습 알고리즘 관련 논문,
양자 선형대수 관련 강의록, 행렬이론 일반 관련 교재를 학습하며 부딪히는
구체적인 질문으로 결정하고자 한다.
\end{document}
