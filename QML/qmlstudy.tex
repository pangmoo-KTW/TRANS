\documentclass[a4paper,atbegshi,chapter]{oblivoir}
\usepackage[dbl4x6]{fapapersize}
\usepackage{amsmath,amssymb,amsfonts,amsthm}
\usepackage{graphicx,xcolor,caption}
\usepackage{braket,hyperref,nicematrix}
\usepackage{enumitem,mdframed}
\usepackage{ob-chapstyles}
\chapterstyle{obchappell}
\setlist{nosep}
\hypersetup{
  colorlinks=true,linkcolor=teal,filecolor=magenta,urlcolor=cyan,
}
\newtheorem{prob}{문제}[chapter]
\newtheorem{exem}{예제}[chapter]
\newtheorem{defn}{정의}[chapter]
\newtheorem{theo}{정리}[chapter]
\newtheorem{thes}{논제}[chapter]
\newtheorem{axio}{공리}[chapter]

\title{\Large 양자선형대수학 학습일지}
\author{계산과학2 노현민-김태원 조}
\date{최초 작성 : 2023년 9월 10일 \\ 최근 편집 : \today}
\begin{document}
\maketitle
\chapter{기초}
\section{양자 데이터}
양자컴퓨터에서 정보는 큐비트로 저장되는데, 큐비트란 복소벡터공간상의
$\ell_2$정규화, 혹은 유클리드 정규화 벡터로 나타낼 수 있는 상태를 뜻한다.
이를테면 $n$ 큐비트로 구성된 상태는 $\sum_x|a_x|^2=1$을 만족하는
$\alpha_x\in\mathbb{C}$에 대해 아래처럼 나타낼 수 있다.
\[
  \ket{\psi}=\sum_{x\in\{0,1\}^n}a_x\ket{x}
\]
양자 상태는 데이터를 추상적으로 저장한다. 이를테면 군{\footnotesize 
group} $G$에 대해 $\ket{g}$를 군의 원소 $g\in G$에 대응하는 기저 상태로
두고 군에 대한 임의의 중첩{\footnotesize superposition}은 아래처럼 나타낼 수
있다. $\ket{g}$가 기저 상태라는 말은 아래 중첩이 유일하다는 뜻이다.
\[
  \ket{\phi}=\sum_{g\in G}b_g\ket{g}
\]
양자 컴퓨터가 상태 $\ket{\psi}$와 상태 $\ket{\phi}$를 저장할 때, 총 상태는
이들 두 상태의 텐서곱으로 주어진다. 이를 아래처럼 표기한다.
\[
  \ket{\psi}\otimes\ket{\phi}=\ket{\psi}\ket{\phi}=\ket{\psi,\phi}
\]
\section{양자 회로}
정규화된 상태를 정규화된 상태로 사상하는{\footnotesize map} 것을
\emph{유니타리{\footnotesize unitary} 연산자} $U$라고 하며 $U$는 아래를 만족한다.
\[
  UU^{\dagger}=U^{\dagger}U=I
\]
$n$ 큐비트에 대한 유니타리 연산자는 원칙상 하나 혹은 두 큐비트 게이트만으로
구현될 수 있다. 따라서 이들 두 큐비트 게이트를 \emph{보편}이라고 한다.

회로는 유니타리 변환에 적절하게 근사해야 한다. 만약 아래 부등식을 정밀도 
$\epsilon$로 만족하는 경우 $U_1,U_2,\ldots,U_t$ 게이트로 구성한 회로가 
$U$를 근사한다고 한다.
\[
  \|U-U_t\ldots U_2U_1\|\leq\epsilon
\]
여기서 $\|\|$는 $\|U-V\|$가 작다면 $U$와 $V$를 그 양자상태와 무관하게 구분하기
어렵다는 성질을 지녀야 하는 적절한 노름으로, 보통 \emph{스펙트럴\footnotesize
spectral} 노름을 사용한다.
\[
  \|A\|:=\max_{\ket{\psi}}\frac{\|A\ket{\psi}\|}{\|\ket{\psi}\|}
\]
여기서 $\|\ket{\psi}\|=\sqrt{\braket{\psi|\psi}}$는 $\ket{\psi}$의 2노름을
나타낸다. 즉 이는 $A$의 최대 특이값{\footnotesize singular value}이다.
\chapter{블록인코딩 개론}
양자컴퓨터 설계의 초기 동기는 양자계{\footnotesize quantum systems}의 
시뮬레이션이다. 이들 시뮬레이션 가운데 가장 간단한 것이 해밀토니언 시뮬레이션이다.
\begin{prob}[해밀토니언 시뮬레이션]
  $H$가 $m$개의 파울리 항으로 구성된 해밀토니언이라고 하자. 다시 말해 파울리
  행렬의 텐서곱 $E_a$에 대해 아래와 같다. 
  \[
    H=\sum_{a=1}^m\lambda_aE_a
  \]
  $\|U-e^{iHt}\|\leq\epsilon$이도록 $e^{-iHt}$와 가까운 유니타리 $U$를 구현하는
  알고리즘을 찾아라.
\end{prob}
해밀토니언을 처음 본다면 $H$를 상호작용하는 항 $E_a$의 합으로 보고 $\lambda_a$를
이런 상호작용의 강도로 보는 편이 좋겠다. 보통 이 문제에 대한 답은 트로터
근사로 주어진다. 충분히 큰 $r$에 대해 아래처럼 근사한다.
\[
  e^{-iHt}\approx(e^{-iE_1t/r}e^{-iE_2t/r}\cdots e^{-iE_mt/r})^r
\]
하지만 이 해는 결코 최적이 아니다. 이 근사의 구현은 $\textrm{poly}(1/\epsilon)$의
게이트복잡도를 요구하기 때문이다. 개선된 알고리즘들은 아래와 같은 프레임워크를
따른다.
\begin{enumerate}[label=(\roman*)]
  \item ``블록인코딩''{\footnotesize block-encoding}이라는 양자회로의 유형을
    정의한다. 
  \item $\lambda_a$와 $E_a$가 주어질 때, $H$의 효율적인 블록인코딩을 구성할
    수 있다는 사실을 보여라.
  \item $H$의 블록인코딩을 적게 사용하면서 $e^{-iHt}$의 근사에 대한 블록인코딩을
    취할 수 있다는 사실을 보여라.
  \item 이런 블록인코딩을 상태의 근사에 적용하라. 
\end{enumerate}
\section{블록인코딩}
\begin{defn} $A\in\mathbb{C}^{r\times c}$가 주어질 때, $U\in\mathbb{C}^{d\times 
  d}$가 $\mathcal{O}(Q)$ 게이트로 구현가능하고 $B_{L,1}\in\mathbb{C}^{d\times r},
  B_{R,1}\in\mathbb{C}^{d\times c}$가 항등행렬의 첫 $r$과 $c$ 열일 때
  아래를 만족하면 $A$의 $Q$블록인코딩이라고 부른다.
  \[
    B^{\dagger}_{L,1}UB_{R,1}=A
  \]
  다시 말해 $U$의 임의의 원소
  $\cdot$ 에 대해 아래와 같다.  
  \[
    U=\begin{pmatrix}A &\cdot\\\cdot &\cdot\end{pmatrix}
  \]
  우리는 $\Pi_L=B_{L,1}B_{L,1}^{\dagger}, \Pi_R=B_{R,1},B_{R,1}^{\dagger}$로 
  $B_{L,1}$과 $B_{R,1}$의 생성에 대한 사영을 표기한다.
\end{defn}
\chapter{해밀토니언 시뮬레이션}
양자계산{\footnotesize quantum computation}이라는 계산 모델{\footnotesize
computational model}의 개념이 아니라 실질적으로 만지고 뜯어볼 수 있는 장치로서
양자컴퓨터가 처음 요구된 이유는 시뮬레이션{\footnotesize simulation}이었다. 
파인만{\footnotesize Richard Ferunman}은 1981년 \snm{컴퓨터로 물리학을 
시뮬레이션해 보자}에서 아래처럼 말한다.
\begin{quote}
  자연은 고전적이지 않단 말이죠. 그러니 빌어먹을, 자연에 대한 시뮬레이션을
  내놓고 싶다면, 양자역학적으로 해야할 것입니다. 이건 세상에나 굉장한 문제인데,
  별로 안 쉬워 보이거든요.
\end{quote}
`양자역학적으로 자연을 시뮬레이션한다'는 말은 곧 양자계{\footnotesize quantum
system}의 동역학{\footnotesize dynamics}을 만져 보겠다는 뜻이다. 그리고 이런
양자계의 동역학, 혹은 양자 상태의 시간(에 따른) 변화를 생성하는 것은 
이른바 \textbf{해밀토니언\footnotesize Hamiltonian} $H$라는 
에르미트\footnote{자기 자신과 켤레 전치가 같은 복소정방해열을 뜻한다. 에르미트라는
수학자가 만들어서 에르미트 행렬이라고 부른다.} 연산자라고 한다. 그리하여
양자역학 시뮬레이션에서 가장 기본적인 문제가 바로 \emph{해밀토니안 
시뮬레이션}이다.
\begin{prob}해밀토니언 $H$, 시간 변화 $t$, 미지의 초기 상태 $\ket{\psi(0)}$에
  대해 최종 상태 $\ket{\psi(t)}$를 (근사적으로) 내놓아라.
\end{prob}
이에 고전컴퓨터는 초기 상태 $\ket{\psi(0)}$도 효과적으로 집어넣을 수 없다. 여기서
시간 $t$에 따른 $H$의 변화는 이상적으로 $e^{-iHt}$이며, 위 문제는 $e^{-iHt}$에
오류 $\epsilon$로 근사하는 유니타리 변환 $U(t)$을 찾으라는 것이다.
\end{document}
