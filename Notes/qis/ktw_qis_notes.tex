\documentclass[a4paper,chapter,atbegshi]{oblivoir}
\usepackage[dbl4x6]{fapapersize}
\usepackage{amsmath,amssymb,amsfonts,amsthm}
\usepackage{graphicx,xcolor,caption}
\usepackage{braket,hyperref,nicematrix}
\usepackage{tikz}
\hypersetup{
  colorlinks=true,linkcolor=teal,filecolor=magenta,urlcolor=cyan,
}
\newtheorem{defn}{정의}[chapter]
\newtheorem{theo}{정리}[chapter]


\title{양자계산복잡도이론 학습일지}
\author{김태원}
\date{최초 작성 : 2023년 8월 27일 \\ 최근 편집 : \today}

\begin{document}
\maketitle
\break
\tableofcontents
\chapter{계산}
\section{대각화}
함수 $f$가 정의역{\tiny domain} $\Delta$상의 원소를 공역{\tiny codomain} 
$\Gamma$상의 원소로 사상{\tiny maps to}한다는 말을 아래처럼 표기한다.
\[
  f:\Delta\rightarrow\Gamma.
\]
$f$의 치역{\tiny range}은 아래와 같다.
\[
  \{f(x)\in\Gamma|x\in\Delta\}.
\]
$f$의 치역이 공역 $\Gamma$와 같다면 $f$는 전사{\tiny surjective}다.
그리고 $\Delta$상의 상이한 원소를 $\Gamma$상의 상이한 원소로 사상하는 $f$는
단사{\tiny injective}다. $f$가 전사이고 단사라면 전단사{\tiny bijcetive}다.

성질{\tiny property} $P$의 \emph{특성함수\tiny characteristic function}
$c_P:\mathbb{N}\rightarrow\{0,1\}$로 $n=P\Rightarrow c_p(n)=0$을 만족한다. 
이때 성질 $P$는 수를 두 집합으로 분할{\tiny partition}한다.

집합 $\Sigma$가 \emph{열거가능\tiny enumerable} 혹은 \emph{가산}이라는
필요충분조건{\tiny iff}은 $\Sigma$가 공집합이거나 전사 함수 $f:\mathbb{N}
\rightarrow\Sigma$가 존재한다는 것이다. 
\begin{theo}
  자연수 순서쌍 $\langle i,j\rangle$의 집합은 가산이다.
\end{theo}
\begin{proof}
  순서쌍을 그림 \ref{fig:11}과 같이 지그재그 꼴의 대각선으로 배열한다.
  그리고
  \[
  \begin{matrix}
     0\mapsto\langle0,0\rangle\quad
     &1\mapsto\langle0,1\rangle\quad
     &3\mapsto\langle0,2\rangle
     &6\mapsto\langle0,3\rangle\\
     &2\mapsto\langle1,0\rangle\quad
     &4\mapsto\langle1,1\rangle
     &7\mapsto\langle1,2\rangle\\
     &&5\mapsto\langle2,0\rangle
     &8\mapsto\langle 2,1\rangle\\
     &&&9\mapsto\langle3,0\rangle
  \end{matrix}
  \quad\cdots
  \]
  와 같이 전단사 $f:\mathbb{N}\rightarrow\mathbb{N}^2$를 정의할 수 있다.
\end{proof}
\begin{figure}[h]
\[
\begin{matrix}
\langle 0,0\rangle &\rightarrow &\langle 0,1\rangle & &\langle0,2\rangle & &\langle 0,3\rangle &\cdots \\
&\swarrow & &\swarrow & &\swarrow & &\swarrow \\
\langle 1,0\rangle & &\langle 1,1\rangle & &\langle 1,2\rangle &&\langle1,3\rangle&\cdots \\
&\swarrow &&\swarrow &&\swarrow && \\
\langle2,0\rangle&&\langle2,1\rangle&&\langle2,2\rangle&&\langle2,3\rangle&\cdots\\
&\swarrow&&\swarrow&&&&\\
\langle3,0\rangle&&\langle3,1\rangle&&\langle3,2\rangle&&\langle3,3\rangle&\cdots\\
\vdots&\swarrow&\vdots&&\vdots&&\vdots&
\end{matrix}
\]
\caption{\label{fig:11}대각 논법}
\end{figure}
\begin{theo}[칸토어 정리(1874)]
  가산이 아닌 무한집합이 존재한다.
\end{theo}
\begin{proof}
  $\mathbb{N}$의 멱집합 $\mathcal{P}$가 아래처럼 존재한다.
  \[
    X\in\mathcal{P}\iff X\subseteq\mathbb{N}.
  \]
  역으로 함수 $f:\mathbb{N}\rightarrow\mathcal{P}$가 존재하여
  $\mathcal{P}$가 가산이라고 하자. 우선 $\mathbb{N}$의 부분집합 $D$를 아래처럼
  둔다.
  \[
    D=\{n\in\mathbb{N}:n\notin f(n)\}.
  \]
  $D\in\mathcal{P}$이고 $f$가 $\mathcal{P}$상의 원소에 대해 가산이기에 어떤
  $d\in\mathbb{N}$가 존재하여 $f(d)=D$를 만족할 것이다. 그리하여 모든 $n\in 
  f(d)$에 대해 아래와 같다. 
  \[
    n\in f(d) \iff n \notin f(n).
  \]
  이는 모순이다. 따라서 $f$와 같은 열거 함수는 존재할 수 없다. 따라서
  멱집합 $\mathcal{P}$는 가산일 수 없다. 즉 비가산{\tiny indenumerable}이다. 
\end{proof}
여기서 $D$를 대각{\tiny diagonal}집합이라고 한다. 대각집합을 직접 사용하지
않고도 대각화의 발상으로 다시 증명할 수도 있다.
\begin{proof}
  무한 이진문자열{\tiny binary strings}의 집합 $\mathbb{B}$가 존재한다.
  역으로 열거 함수 $f:\mathbb{N}\rightarrow\mathbb{B}$가 존재한다고 가정하자.
  \[
    \begin{matrix}
      0\rightarrow b_0:\underline{0}110001010011\ldots\\
      1\rightarrow b_1:1\underline{1}00101001101\ldots\\
      2\rightarrow b_2:11\underline{0}0101100001\ldots\\
      \vdots
    \end{matrix}
  \]
  대각선을 따라 $n\in\mathbb{N}$을 $n$번째 문자열 $b_n\in\mathbb{B}$의 $n+1$번째
  자릿수{\tiny digit}로 사상하는 것이다. 그리고 이제 그 $n$번째 자릿수에 대해
  $0$과 $1$을 뒤바꾼다. 이렇게 대각자릿수를 뒤집은 문자열 $d$는 $b_0$과 $1$번째
  자릿수에 대해 다르고, $b_1$은 $2$번째 자릿수에 대해 다르고, $b_2$는 $3$번째
  자릿수에 대해 다르다. 따라서 모든 $n\in\mathbb{N}$을 $b_n\in\mathbb{B}$에
  대해 사상한 집합은 $d\in\mathbb{B}$를 포함하지 않는다. 그리하여 $f:\mathbb{N}
  \rightarrow\mathbb{B}$는 열거함수가 아니다. 이는 모순이다. 따라서 모든
  함수 $f:\mathbb{N}\rightarrow\mathbb{B}$는 열거함수일 수 없고, 따라서
  $\mathbb{B}$는 비가산이다.
\end{proof}
\chapter{튜링}
\end{document}
