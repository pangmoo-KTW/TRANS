\documentclass[a4paper,chapter,atbegshi]{oblivoir}
\usepackage[dbl4x6]{fapapersize}
\usepackage{amsmath,amssymb,amsfonts,amsthm}
\usepackage{graphicx,xcolor,caption}
\usepackage{braket,hyperref,nicematrix}
\usepackage{tikz,fourier}
\hypersetup{
  colorlinks=true,linkcolor=teal,filecolor=magenta,urlcolor=cyan,
}
\newtheorem{defn}{정의}[chapter]
\newtheorem{theo}{정리}[chapter]
\newtheorem{thes}{논제}[chapter]

\title{양자계산복잡도이론 학습일지}
\author{김태원}
\date{최초 작성 : 2023년 8월 27일 \\ 최근 편집 : \today}

\begin{document}
\maketitle
\break
\tableofcontents
\chapter{계산}
\section{대각화}
함수 $f$가 정의역{\tiny domain} $\Delta$상의 원소를 공역{\tiny codomain} 
$\Gamma$상의 원소로 사상{\tiny maps to}한다는 말을 아래처럼 표기한다.
\[
  f:\Delta\rightarrow\Gamma.
\]
$f$의 치역{\tiny range}은 아래와 같다.
\[
  \{f(x)\in\Gamma|x\in\Delta\}.
\]
$f$의 치역이 공역 $\Gamma$와 같다면 $f$는 전사{\tiny surjective}다.
그리고 $\Delta$상의 상이한 원소를 $\Gamma$상의 상이한 원소로 사상하는 $f$는
단사{\tiny injective}다. $f$가 전사이고 단사라면 전단사{\tiny bijcetive}다.

성질{\tiny property} $P$의 \emph{특성함수\tiny characteristic function}
$c_P:\mathbb{N}\rightarrow\{0,1\}$로 $n=P\Rightarrow c_p(n)=0$을 만족한다. 
이때 성질 $P$는 수를 두 집합으로 분할{\tiny partition}한다.

집합 $\Sigma$가 \emph{열거가능\tiny enumerable} 혹은 \emph{가산}이라는
필요충분조건{\tiny iff}은 $\Sigma$가 공집합이거나 전사 함수 $f:\mathbb{N}
\rightarrow\Sigma$가 존재한다는 것이다. 
\begin{theo}\label{theo:11}
  자연수 순서쌍 $\langle i,j\rangle$의 집합은 가산이다.
\end{theo}
\begin{proof}
  순서쌍을 그림 \ref{fig:11}과 같이 지그재그 꼴의 대각선으로 배열한다.
  그리고
  \[
  \begin{matrix}
     0\mapsto\langle0,0\rangle\quad
     &1\mapsto\langle0,1\rangle\quad
     &3\mapsto\langle0,2\rangle
     &6\mapsto\langle0,3\rangle\\
     &2\mapsto\langle1,0\rangle\quad
     &4\mapsto\langle1,1\rangle
     &7\mapsto\langle1,2\rangle\\
     &&5\mapsto\langle2,0\rangle
     &8\mapsto\langle 2,1\rangle\\
     &&&9\mapsto\langle3,0\rangle
  \end{matrix}
  \quad\cdots
  \]
  와 같이 전단사 $f:\mathbb{N}\rightarrow\mathbb{N}^2$를 정의할 수 있다.
\end{proof}
\begin{figure}[h]
\[
\begin{matrix}
\langle 0,0\rangle &\rightarrow &\langle 0,1\rangle & &\langle0,2\rangle & &\langle 0,3\rangle &\cdots \\
&\swarrow & &\swarrow & &\swarrow & &\swarrow \\
\langle 1,0\rangle & &\langle 1,1\rangle & &\langle 1,2\rangle &&\langle1,3\rangle&\cdots \\
&\swarrow &&\swarrow &&\swarrow && \\
\langle2,0\rangle&&\langle2,1\rangle&&\langle2,2\rangle&&\langle2,3\rangle&\cdots\\
&\swarrow&&\swarrow&&&&\\
\langle3,0\rangle&&\langle3,1\rangle&&\langle3,2\rangle&&\langle3,3\rangle&\cdots\\
\vdots&\swarrow&\vdots&&\vdots&&\vdots&
\end{matrix}
\]
\caption{\label{fig:11}대각 논법}
\end{figure}
\begin{theo}[칸토어 정리(1874)]\label{theo:12}
  가산이 아닌 무한집합이 존재한다.
\end{theo}
\begin{proof}
  $\mathbb{N}$의 멱집합 $\mathcal{P}$가 아래처럼 존재한다.
  \[
    X\in\mathcal{P}\iff X\subseteq\mathbb{N}.
  \]
  역으로 함수 $f:\mathbb{N}\rightarrow\mathcal{P}$가 존재하여
  $\mathcal{P}$가 가산이라고 하자. 우선 $\mathbb{N}$의 부분집합 $D$를 아래처럼
  둔다.
  \[
    D=\{n\in\mathbb{N}:n\notin f(n)\}.
  \]
  $D\in\mathcal{P}$이고 $f$가 $\mathcal{P}$상의 원소에 대해 가산이기에 어떤
  $d\in\mathbb{N}$가 존재하여 $f(d)=D$를 만족할 것이다. 그리하여 모든 $n\in 
  f(d)$에 대해 아래와 같다. 
  \[
    n\in f(d) \iff n \notin f(n).
  \]
  이는 모순이다. 따라서 $f$와 같은 열거 함수는 존재할 수 없다. 따라서
  멱집합 $\mathcal{P}$는 가산일 수 없다. 즉 비가산{\tiny indenumerable}이다. 
\end{proof}
여기서 $D$를 대각{\tiny diagonal}집합이라고 한다. 대각집합을 직접 사용하지
않더라도 대각화라는 발상만으로 다시 증명할 수도 있다.
\begin{proof}
  무한 이진문자열{\tiny binary strings}의 집합 $\mathbb{B}$가 존재한다.
  역으로 열거 함수 $f:\mathbb{N}\rightarrow\mathbb{B}$가 존재한다고 가정하자.
  \[
    \begin{matrix}
      0\rightarrow b_0:\underline{0}110001010011\ldots\\
      1\rightarrow b_1:1\underline{1}00101001101\ldots\\
      2\rightarrow b_2:11\underline{0}0101100001\ldots\\
      \vdots
    \end{matrix}
  \]
  대각선을 따라 $n\in\mathbb{N}$을 $n$번째 문자열 $b_n\in\mathbb{B}$의 $n+1$번째
  자릿수{\tiny digit}로 사상하는 것이다. 그리고 이제 그 $n$번째 자릿수에 대해
  $0$과 $1$을 뒤바꾼다. 이렇게 대각자릿수를 뒤집은 문자열 $d$는 $b_0$과 $1$번째
  자릿수에 대해 다르고, $b_1$은 $2$번째 자릿수에 대해 다르고, $b_2$는 $3$번째
  자릿수에 대해 다르다. 따라서 모든 $n\in\mathbb{N}$을 $b_n\in\mathbb{B}$에
  대해 사상한 집합은 $d\in\mathbb{B}$를 포함하지 않는다. 그리하여 $f:\mathbb{N}
  \rightarrow\mathbb{B}$는 열거함수가 아니다. 이는 모순이다. 따라서 
  열거함수 $f:\mathbb{N}\rightarrow\mathbb{B}$는 존재하지 않는다. 다시 말해
  $\mathbb{B}$는 비가산이다.
\end{proof}
\section{계산가능성}
알고리즘을 부분함수에 대한 계산{\tiny computation}으로 정의할 수 있다. 
부분함수{\tiny partial function}란 정의역상의 인자{\tiny argument}에 대해
출력이 존재하지 않을 수도 있는 사상 $f$다.

또한 계산가능성{\tiny computability}을 계산기{\tiny computer}의 크기나 속도와
완전히 무관하게 정의할 수 있다. 이처럼 급진적인 추상화는 알고리즘 계산의
가능성과 한계에 대해 내놓을 수 있는 모든 주장을 강화한다. 계산 모델로는
튜링장치{\tiny Turing Machine}가 있을 수 있다.
\begin{thes}[튜링 1936]
  비형식적으로 말해 효과적으로 계산가능한 수치 함수는 실상 전부 적절한 튜링장치로
  계산가능한 함수들이다.
\end{thes}
튜링 논제에서 유의해야 하는 표현은 \emph{수치함수\tiny numerical function}다. 
모든 비수치{\tiny nonnumerical} 객체 $X$를 어떤 수로 사상 혹은 부호화할 수 있다.
이런 $X$를 표준형식언어{\tiny standard formal languages}상의 
\emph{표현식\tiny expressions}이라고 부른다.
\begin{defn}
  수치적인 성질 혹은 관계가 \underline{효과적으로 결정가능\tiny effectively
  decidable}하다는 말의 필요충분조건{\tiny iff}은 그 특성함수가 효과적으로
  계산가능하다는 것이다.
\end{defn}
\begin{defn}
  집합 $\Sigma$가 \underline{효과적으로 결정가능}하다는 말의 필요충분조건{\tiny
  iff}은 $\Sigma$의 성질에 대한 특성함수 $c_{\Sigma}$가 효과적으로 계산가능하다는
  것이다.
\end{defn}
\begin{theo}
  모든 자연수 집합은 효과적으로 결정가능하다.
\end{theo}
\begin{theo}\label{theo:14}
  $\Sigma$가 효과적으로 결정가능한 집합이라면 그 여집합{\tiny complement}
  $\overline{\Sigma}$도 효과적으로 결정가능한 집합이다.
\end{theo}
\begin{proof}
  $\Sigma\subseteq\mathbb{N}$이 유한하다면 특성함수 $c_{\Sigma}$는 항상 $1$이나
  $0$의 값을 지닌다. 이런 함수는 무차별대입{\tiny brute-force} 알고리즘으로
  $1$이나 $0$을 계산가능하다. 따라서 모든 자연수 집합은 효과적으로 결정가능하다.

  또한 특성함수 $c_{\Sigma}$가 효과적으로 계산가능하다면 $\overline{\Sigma}$의
  특성함수 $\overline{c}$는 다음처럼 정의될 수 있다.
  \[
    \overline{c}(n) = 1-c_{\Sigma}(n)
  \]
  그리고 $c_{\Sigma}$의 계산가능성으로 $\overline{c}$ 또한 계산가능하다.
\end{proof}
\begin{defn}
  집합 $\Sigma$는 \underline{효과적으로 열거가능\tiny effectively 
  enumerable}하다는 말의 필요충분조건은 $\Sigma$가 공집합이거나 
  효과적인 계산가능 함수가 존재하여 $\Sigma$를 열겨한다는 것이다.
\end{defn}
\begin{theo}\label{theo:15}
  $\Sigma$가 효과적으로 결정가능한 집합이라면 효과적으로 열거가능하다.
\end{theo}
\begin{proof}
  $s\in\Sigma$가 있다고 하자. 그리고 입력 $n$에 대해 $n\in\Sigma$를 효과적으로
  확인할 수 있는 알고리즘에 대해 $n\in\Sigma$라면 $n$을 출력하고
  $n\notin\Sigma$라면 $s$를 출력한다고 하자. 이 알고리즘은 전사함수
  $f:\mathbb{N}\rightarrow\Sigma$를 계산한다. 따라서 $\Sigma$는 효과적으로
  열거가능하다.
\end{proof}
\begin{theo}
  $\Sigma$와 그 여집합 $\overline{\Sigma}$가 모두 효과적으로 열거가능한
  집합이라면 $\Sigma$는 효과적으로 결정가능하다.
\end{theo}
\begin{proof}
  $\Sigma$가 계산가능함수 $f$에 의해 열거가능한 집합이며 $\overline{\Sigma}$가
  계산가능함수 $g$에 의해 열거가능한 집합이라고 하자. 차례로 계산한다.
  \[
    f(0),g(0),f(1),g(1),f(2),g(2),\ldots
  \]
  임의로 주어진 $s$에 대해 $s\in\Sigma\iff s\notin\overline{\Sigma}$ 혹은
  $s\in\overline{\Sigma}\iff s\notin\Sigma$이므로
  $s$는 무조건 출력된다. 다시 말해 어떤 $m$이 존재하여 $f(m)=s$를 만족하거나
  어떤 $n$이 존재하여 $g(n)=s$를 만족한다. 따라서 $f$ 혹은 $g$는 효과적으로
  계산가능하기에 $\Sigma$ 혹은 $\overline{\Sigma}$가 결정가능한 집합이다. 그리고
  $\overline{\overline{\Sigma}}=\Sigma$이므로 $\Sigma$는 효과적으로 결정가능하다.
\end{proof}
이쯤 알고리즘을 더 명확하게 정의한다.
\begin{defn}
  알고리즘의 정의역은 입력 $n\in\mathbb{N}$에 대해 알고리즘 실행이 언젠가는
  종결하고 어떤 수를 출력으로 내놓는 자연수의 집합이다. 
\end{defn}
\begin{theo}
  $W$가 효과적으로 열거가능한 집합인 필요충분조건은 $W$가 어떤 알고리즘의
  정의역이라는 것이다.
\end{theo}
\begin{proof}
  ($\Rightarrow$)
  $W$가 열거가능한 집합이라고 하자. 그렇다면 정의상 (i) $W$가 공집합이거나
  (ii) 계산가능함수 $f$가 존재하여 $W$를 열거한다.

  (i)의 경우 아무 출력도 내놓지 않는 알고리즘 아무거나 고르면 된다.
  (ii)의 경우 어떤 알고리즘 $\Pi$가 존재하여 함수 $f$를 계산한다. $\Pi$로
  더 복잡한 알고리즘 $\Pi^+$를 구성할 수 있다. 주어진 입력 $n$에 대해 $\Pi$로
  루프{\tiny loop}하며 $f(0),f(1),f(2),\ldots$를 계산하다가 어떤 $i$에 대해
  $f(i)=n$인 경우 멈추고 $i$를 출력한다. 따라서 $\Pi^+$의 정의역은
  $W$다. 

  ($\Leftarrow$) $W$가 어떤 알고리즘 $\Pi$의 정의역이라고 하자. $W$가 공집합이라면
  $W$는 열거가능하다. $W$가 공집합이 아니라고 하자. 정리 \ref{theo:11}에 의해
  각각의 가능한 쌍 $\langle i,j\rangle$에 대해 $n\in\mathbb{N}$과 일대일대응이
  존재한다. 그리고 이에 계산가능함수 $\textrm{fst}(n)$과 $\textrm{snd}(n)$이
  존재하여 각각 $n$번째 쌍의 첫 번째 성분 $i$와 두 번째 성분 $j$를 반환한다. 
  이들 함수로 $\Pi'$를 다음처럼 정의한다.

  주어진 입력 $n$에 대해 $i=\textrm{fst}(n)$과 $j=\textrm{snd}(n)$를 계산한다.
  그리고 $\Pi$를 입력 $i$에 대해 $j$번 실행한다. $\Pi$가 입력 $i$에
  대해 어떤 출력 $j$로 정지{\tiny halt}하면 $\Pi'$는 $i$를 출력한다. 
  그 외의 경우 $\Pi'$는 $o$를 출력한다.

  $n$이 증가할 때마다 $\Pi'$는 모든 $i,j$에 대해 $\Pi$를 확인한다. 그리고 
  $\Pi$가 결국 어떤 출력을 내놓는 $i$를 출력한다. 그리하여 $\Pi'$는
  $\Pi$의 정의역 $W$를 치역으로 지니는 함수를 계산한다. 따라서 $W$는
  열거가능하다.
\end{proof}
\begin{theo}
  모든 효과적으로 열거가능한 자연수 집합들의 집합 $\mathcal{W}$는 열거가능하다.
\end{theo}
당연하다. 이는 아래 같은 따름정리를 유도한다.
\begin{theo}
  어떤 집합은 효과적으로 열거가능하지 않기에 효과적으로 결정가능하지 않다.
\end{theo}
\begin{proof}
  정리 \ref{theo:12}에 의해 $\mathbb{N}$의 멱집합 $\mathcal{P}$는 열거가능하지
  않다. 따라서 $\mathcal{W}\neq\mathcal{P}$다. 그러나 $\mathcal{W}\subset
  \mathcal{P}$다. 그래서 $\mathcal{W}$에 없는 것들이 $\mathcal{P}$에는 존재한다.
  즉 효과적으로 열거가능하지 않은 집합들이 존재하는 것이다.
  정리 \ref{theo:15}의 대우에 의해 이들 집합은 결정불가능하다.
\end{proof}
\begin{theo}[열거가능집합의 근본 정리]
  효과적으로 열거가능한 집합 $K$가 존재하여 여집합 $\overline{K}$는
  효과적으로 열거불가능하다.
\end{theo}
\begin{proof}
  $K$를 아래처럼 정의한다.
  \[
    K:=\{e|e\in W_e\}.
  \]
  정의상 모든 $e$에 대해 아래가 성립한다.
  \[
    e\in\overline{K}\iff e\notin W_e.
  \]
  그래서 $\overline{K}$는 모든 $W_e$에 대해 다르다. 그래서 $\overline{K}$는
  효과적으로 열거가능한 집합이 아니다. $W_e$가 전부이기 때문이다.

  $\overline{K}$가 효과적으로 열거불가능하기에 $\overline{K}$는 $\mathbb{N}$
  전체일 수 없다. 그리하여 $K$는 공이 아닌 집합이다. $o$를 $K$상의 어떤
  원소로 두고 알고리즘 $\Pi''$를 아래처럼 정의한다.
  \begin{quote}
    주어진 입력 $n$에 대해 $i=\textrm{fst}(n),j=\textrm{snd}(n)$을 계산한다.
    그리고 알고리즘 $\Pi_i$를 찾아 입력 $i$에 대해 $j$번 실행한다.
    $\Pi_i$가 입력 $i$에 대해 $j$로 출력하며 정지한다면
    $\Pi''$는 $i$를 출력한다. 이외의 경우 $\Pi''$는 기본값 $o$를
    출력한다.
  \end{quote}
  $n$이 증가하며 $\Pi''$는 모든 쌍 $\langle i,j\rangle$에 대해 실행한다.
  그래서 $\Pi''$의 출력은 $i$가 $\Pi_i$의 정의역인 모든 $i$의 집합이다.
  즉 $i\in W_i$이며 다시 말해 $K$다. 따라서 $K$는 효과적으로 열거가능하다.
\end{proof}
\begin{theo}
  어떤 효과적으로 열거가능한 집합은 결정가능하지 않다.
\end{theo}
\begin{proof}
  열거불가능한 여집합을 지니는 임의의 열거가능 집합 $K$를 취한다. $K$가
  결정가능하다고 가정하자.
  정리 \ref{theo:14}에 의해 $K$가 결정가능하면 그 여집합 $\overline{K}$ 또한
  결정가능할 것이다. 하지만 그렇다면 정리 \ref{theo:15}에 의해 $\overline{K}$는
  열거가능 집합일 것이다. 이는 모순이다. 따라서 어떤 열거가능 집합은 
  결정불가능하다.
\end{proof}
\chapter{튜링}
\end{document}
