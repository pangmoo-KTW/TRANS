\documentclass[atbegshi, chapter]{memoir}
\usepackage{fapapersize}
\usefapapersize{*,*,30mm,*,30mm,*}
\usepackage{amsmath,amssymb,amsfonts,amsthm}
\usepackage{hyperref,enumitem,mdframed,euler}
\hypersetup{colorlinks=true, linkcolor=magenta,}
\chapterstyle{chappell}
\setlist{nosep}
\newtheorem{thm}{Theorem}[chapter]
\newtheorem{lem}[thm]{Lemma}
\newtheorem{exm}[thm]{Example}
\newtheorem*{cor}{Corollary}
\newtheorem*{WOP}{Well-Ordering Principle}
\title{Elementary Number Theory}
\author{KTW}
\date{\today}
\begin{document}
\maketitle\thispagestyle{empty}\newpage
\tableofcontents
\chapter{Preliminaries}
\section{Mathematical Induction}
\begin{WOP}
  Every nonempty set $S$ of nonnegative integers contains a least element;
  that is, there is some integer $a$ in $S$ such that $a\leq b$ for all 
  $b$'s belonging to $S$.
  \[
    \forall S \subseteq\mathbb{Z}_{>0}, \exists a\in S:\forall b\in S, a \leq b
  \]
\end{WOP}
\begin{thm}[Archimedian Property]
  If $a$ and $b$ are any positive integers, then there exists a positive
  integer $n$ such that $na\geq b$.
  \[
    \forall a,b\in\mathbb{Z}_{>0}, \exists n\in\mathbb{Z}_{>0}:na\geq b
  \]
\end{thm}
\begin{proof}
  For reductio ad aubsurdum, suppose that the statement of the therorem is
  false. Then:
  \begin{align*}
  &\neg(\forall a,b\in\mathbb{Z}_{>0},\exists n\in\mathbb{Z}_{>0}:na\geq b)\\
    \Leftrightarrow\;&\exists a,b\in\mathbb{Z}_{>0}: \forall n\in\mathbb{Z}_{>0}, na<b\\
  \Leftrightarrow\;&\exists a,b\in\mathbb{Z}_{>0}: \forall n\in\mathbb{Z}_{>0},0<b-na\\
  \Leftrightarrow\;&S=\{b-na\;|\;n\in\mathbb{Z}_{>0}\} = \mathbb{Z}_{>0}\\
  \Rightarrow\;&\exists b-ma\in S:\forall b-na\in S, b-ma\leq b-na
  \quad[\textrm{by the Well-Ordering Principle}]\\
  \Rightarrow\;& \exists b-ma, b-(m+1)a\in S:\forall b-na\in S,
  b-(m+1)a < b-ma\leq b-na \longrightarrow\bot \\
  \Rightarrow\;&\forall a,b\in\mathbb{Z}_{>0},\exists n\in\mathbb{Z}_{>0}:na\geq b
  \end{align*}
\end{proof}
\begin{thm}[First Principle of Finite Induction]
  Let $S$ be a set of positive integers with the following properties:
  \begin{enumerate}[label=(\alph*)]
    \item $\exists 1\in S$
    \item $\forall k\in S, \exists k+1\in S$
  \end{enumerate}
  Then $S$ is the set of all positive integers.
\end{thm}
\begin{proof}
  Let $S$ be a set of positive integers such that:
  \[
    \exists 1\in S\;\&\;\forall k\in S, \exists k+1\in S
  \]
  For reductio ad absurdum, 
  let $T$ be a nonempty set of all positive integers not in $S$, that is:
  \[
    T=\{t\in\mathbb{Z}_{>0}\;|\;t\notin S\}
  \]
  By the Well-Ordering Principle, $T$ has a least element $a$, and:
  \begin{align*}
    1\in S &\Rightarrow 1<a\in T \\
           &\Leftrightarrow 0<a-1\notin T \\
           &\Leftrightarrow a-1\in S\\
           &\Leftrightarrow a\in S\longrightarrow\bot\\
           &\Rightarrow T=\varnothing\\
           &\Leftrightarrow S = \mathbb{Z}_{>0}
  \end{align*}
\end{proof}
\begin{thm}[Second Principle of Finite Induction]
  Let $S$ be a set of positive integers with the following properties:
  \begin{enumerate}[label=(\alph*)]
    \item $\exists1\in S$
    \item $\exists 1,2,\ldots, k\in S\Rightarrow \exists k+1\in S$
  \end{enumerate}
  Then $S$ is the set of all positive integers.
\end{thm}
\begin{proof}
  Let $S$ be a set of positive integers follwing properties above.
  For reductio ad absurdum, let $T$ be a nonempty set of all positive integers
  not in $S$. By the Well-Ordering Principle, $T$ has a least element $a$, and:
  \begin{align*}
    1\in S &\Rightarrow 1<a\in T \\
           &\Leftrightarrow 0<1,\ldots,a-1\notin T\\
           &\Leftrightarrow 1,\ldots,a-1\in S \\
           &\Leftrightarrow a\in S \longrightarrow\bot\\
           &\Rightarrow T = \varnothing \\
           &\Leftrightarrow S = \mathbb{Z}_{>0}
  \end{align*}
\end{proof}
\begin{exm}[Lucas sequence]\normalfont
  \[
    1,3,4,7,11,18,29,47,76,\ldots
  \]
  Sequence above may be defined inductively by
  \[
    \begin{cases}
      a_1 = 1 \\
      a_2 = 3 \\
      a_n = a_{n-1}+a_{n-2}\quad\textrm{ for all }n\geq 3
    \end{cases}
  \]
  We contend that the inequality
  \[
    a_n < \left(\frac{7}{4}\right)^n
  \]
  holds for every positive integer $n$. First of all, for $n=1$ and $2$, we
  have
  \[
    a_1 = 1 < \left(\frac{7}{4}\right)^1\quad\&\quad
    a_2 = 3 < \left(\frac{7}{4}\right)^2 = 3\frac{1}{16}
  \]
  and this provides a basis for the induction. For the induction step,
  choose an integer $k\geq3$ and assume that the inequality is valid
  for $n=1,2,\ldots,k-1$. Then, in particular:
  \[
    a_{k-1}<\left(\frac{7}{4}\right)^{k-1}\quad\&\quad
    a_{k-2}<\left(\frac{7}{4}\right)^{k-2}
  \]
  By the way in which the sequence is formed, it follows that:
  \begin{align*}
    a_k = a_{k-1}+a_{k-2}
    &< \left(\frac{7}{4}\right)^{k-1}+\left(\frac{7}{4}\right)^{k-2} \\
    &= \left(\frac{7}{4}\right)^{k-2}\left(\frac{7}{4}+1\right)\\
    &=\left(\frac{7}{4}\right)^{k-2}\left(\frac{11}{7}\right)\\
    &<\left(\frac{7}{4}\right)^{k-2}\left(\frac{7}{4}\right)^{2}
    =\left(\frac{7}{4}\right)^{k}
  \end{align*}
  Because the inequality is true for $n=k$ whenever it is true for the
  integers $1,2,\ldots,k-1$, we conclude by the second induction principle
  that $a_n<(7/4)^n$ for all $n\geq 1$.
\end{exm}
\section*{Problems 1.1}
\textbf{1. }Establish the formulas below by mathamatical induction:
\begin{enumerate}[label=(\alph*)]
  \item $1+2+3+\cdots+n=\frac{n(n+1)}{2}$ for all $n\geq 1.$
  \item $1+3+5+\cdots+(2n-1)=n^2$ for all $n\geq 1$.
  \item $1\cdot 2 + 2\cdot 3 + 3\cdot 4 + \cdots +n(n+1)=\frac{n(n+1)(n+2)}{3}$
    for all $n\geq 1$.
  \item $1^2+3^2+5^2+\cdots+(2n-1)^2=\frac{n(2n-1)(2n+1)}{3}$
    for all $n\geq 1$.
  \item $1^3+2^3+3^3+\cdots +n^3 = \left[\frac{n(n+1)}{2}\right]^2$
    for all $n\geq 1.$
\end{enumerate}
\hfill\break
$\pmb{\sharp 1(a).}$ Let $S$ be a set such that:
\[
  S = \bigg\lbrace n\in\mathbb{N}:1+2+3+\cdots+n = \frac{n(n+1)}{2}\bigg\rbrace
\]
$1\in S$ because for $n=1$:
\[
  1 = \frac{1(1+1)}{2}
\]
Let any $k\in\mathbb{N}$ be the member of $S$, that is:
\[
  1+2+3+\cdots+k = \frac{k(k+1)}{2}
\]
Then;
\begin{align*}
  1+2+3+\cdots+k+(k+1)&=\frac{k(k+1)}{2}+(k+1) \\
                   &=\frac{k(k+1)}{2}+\frac{2(k+1)}{2} \\
                   &=\frac{(k+1)(k+2)}{2}
\end{align*}
therefore by the First Principle of Finite Induction, 
$S=\mathbb{N}$.\hfill$\blacksquare$

\hfill\break
$\pmb{\sharp 1(b).}$ Let $S$ be a set such that:
\[
  S = \bigg\lbrace n\in\mathbb{N}: 1+3+5+\cdots+(2n-1)=n^2\bigg\rbrace
\]
$1\in S$ because for $n=1$:
\[
  (2\cdot 1-1)=1=1^2
\]
Let any $k\in\mathbb{N}$ be the member of $S$, that is:
\[
  1+3+5+\cdots+(2k-1) = k^2
\]
Then;
\begin{align*}
  1+3+\cdots+(2k-1)+[2(k+1)-1] &= k^2+[2(k+1)-1] \\
                               &= k^2+2k+1 \\
                               &= (k+1)^2
\end{align*}
therefore by the First Principle of Finite Induction, $S=\mathbb{N}$.
\hfill$\blacksquare$

\hfill\break
$\pmb{\sharp 1(c).}$ Let $S$ be a set such that:
\[
  S = \bigg\lbrace n\in\mathbb{N}:1\cdot 2+2\cdot3 +\cdots + n(n+1)
  =\frac{n(n+1)(n+2)}{3}\bigg\rbrace
\]
$1\in S$ because for $n=1$:
\[
  1(1+1) = 2 = \frac{1(1+1)(1+2)}{3}
\]
Let any $k\in\mathbb{N}$ be the member of $S$, that is:
\[
  1\cdot2+ 2\cdot3 +\cdots+k(k+1)=\frac{k(k+1)(k+2)}{3}
\]
Then;
\begin{align*}
  1\cdot2+2\cdot3+\cdots+k(k+1)+(k+1)(k+2)&=\frac{k(k+1)(k+2)}{3}+(k+1)(k+2)\\
      &=\frac{k(k+1)(k+2)}{3}+\frac{3(k+1)(k+2)}{3} \\
      &=\frac{(k+1)(k+2)(k+3)}{3}
\end{align*}
therefore by the First Principle of Finite Induction, $S=\mathbb{N}$.

\hfill\break
$\pmb{\sharp 1(d).}$ Let $S$ be a set such that:
\[
  S = \bigg\lbrace n\in\mathbb{N}:1^2+3^2+5^2+\cdots+(2n-1)^2=
    \frac{n(2n-1)(2n+1)}{3}
\]
$1\in S$ because for $n=1$:
\[
  (2\cdot1-1)^2 = 1 = \frac{3}{3}
\]
Let any $k\in\mathbb{N}$ be the member of $S$, that is:
\[
  1^2+3^2+\cdots+(2k-1)^2=\frac{k(2k-1)(2k+1)}{3}
\]
Then;
\begin{align*}
  1^2+3^2+\cdots+(2k-1)^2+(2k+1)^2
  &= \frac{k(2k-1)(2k+1)}{3}+(2k+1)^2 \\
  &= \frac{k(2k-1)(2k+1)}{3}+\frac{3(2k+1)^2}{3} \\
  &= \frac{(2k+1)[k(2k-1)+3(2k+1)]}{3} \\
  &= \frac{(2k+1)[2k^2+5k+3]}{3} \\
  &= \frac{(2k+1)(2k+3)(k+1)}{3}\\
  &= \frac{(k+1)(2k+1)(2k+3)}{3}
\end{align*}
therefore by the First Principle of Finite Induction, 
$S=\mathbb{N}$.\hfill$\blacksquare$

\hfill\break
$\pmb{\sharp 1(e).}$ Let $S$ be a set such that:
\[
  S = \bigg\lbrace n\in\mathbb{N}: 1^3+2^3+\cdots+n^3=
    \left[\frac{n(n+1)}{2}\right]^2\bigg\rbrace
\]
$1\in S$ for $n=1$ beacuase:
\[
  1^3 = 1 = 1^2
\]
Let any $k\in\mathbb{N}$ be the member of $S$, that is:
\[
  1^3 + 2^3+ \cdots +k^3 = \left[\frac{k(k+1)}{2}\right]^2
\]
Then;
\begin{align*}
  1^3+2^3+\cdots+k^3+(k+1)^3
  &= \left[\frac{k(k+1)}{2}\right]^2+(k+1)^3 \\
  &= \frac{k^2}{4}(k+1)(k+1)+(k+1)(k+1)(k+1) \\
  &=(k+1)^2\left(\frac{k^2}{4}+(k+1)\right) \\
  &=(k+1)^2\left(\frac{k}{2}+1\right)^2\\
  &=(k+1)^2\left(\frac{k+2}{2}\right)\\
  &=\left[\frac{(k+1)(k+2)}{2}\right]^2
\end{align*}
therefore by the First Principle of Finite Induction, 
$S=\mathbb{N}.$\hfill$\blacksquare$

\hfill\break
\textbf{2. }If $r\neq1$, show that for any positive integer $n$,
\[
  a + ar + ar^2 + \cdots + ar^n = \frac{a(r^{n+1}-1}{r-1}
\]

\hfill\break
$\pmb{\sharp 2.}$ Let $S$ be a set such that:
\[
  S = \bigg\lbrace n\in\mathbb{N}: a+ar+ar^2+\cdots+ar^n=
    \frac{a(r^{n+1}-1)}{r-1}
\]
$1\in S$ for $n=1$ because:
\[
  a+ar^1 =a(r+1)=\frac{a(r+1)(r-1)}{r-1} =\frac{a(r^2-1)}{r-1}
\]
Let any $k\in\mathbb{N}$ be the member of $S$, that is:
\[
  a + ar+\cdots+ar^k=\frac{a(r^{k+1}-1)}{r-1}
\]
Then;
\begin{align*}
  a + ar +\cdots+ar^k+ar^{k+1}
  &=\frac{a(r^{k+1}-1)}{r-1}+ar^{k+1}\\
  &=\frac{a(r^{k+1}-1)}{r-1}+\frac{(r-1)ar^{k+1}}{r-1}\\
  &=\frac{ar^{k+2}-ar^{k+1}+ar^{k+1}-a}{r-1}\\
  &=\frac{ar^{k+2}-a}{r-1}\\
  &=\frac{a(r^{k+2}-1)}{r-1}
\end{align*}
therefore by the First Principle of Finite Induction,
$S=\mathbb{N}$.\hfill$\blacksquare$

\hfill\break
\textbf{3. }Use the Second Principle of Finite Induction to establish that
for all $n\geq1$,
\[
  a^n-1=(a-1)(a^{n-1}+a^{n-2}+\cdots+a+1)
\]
[Hint: $a^{n+1}-1=(a+1)(a^n-1)-a(a^{n-1}-1)$.]

\hfill\break
$\pmb{\sharp 3.}$ Let $S$ be a set such that:
\[
  S = \big\lbrace n\in\mathbb{N}: a^n-1 = (a-1)(a^{n-1}+a^{n-2}+\cdots+
    a+1\big\rbrace
\]
$1\in S$ for $n=1$ because:
\[
  a-1 = (a-1)a^0
\]
Let any $k\in\mathbb{N}$ be the member of $S$, that is:
\[
  a^k-1 = (a-1)(a^{k-1}+a^{k-2}+\cdots+a+1)
\]
Then;
\begin{align*}
  (a+1)(a^k-1)-a(a^{k-1}-1)
  &=(a+1)(a-1)(a^{k-1}+a^{k-2}+\cdots+a+1)-a(a^{k-1}-1) \\
  &=(a-1)(a+1)(a^{k-1}+a^{k-2}+\cdots+a+1)-a(a^{k-1}-1) \\
  &=(a-1)[a(a^{k-1}+\cdots+a+1)+(a^{k-1}+\cdots+a+1)]-a(a^{k-1}-1) \\
  &=(a-1)[a^k+2(a^{k-1}+\cdots+a)+1]-a(a^{k-1}-1)\\
  &=(a-1)[a^{k}+2(a^{k-1}+\cdots+a)+1]-a^k+a
\end{align*}
since
\[
  a^k=(a-1)(a^{k-1}+a^{k-2}+\cdots+a)+a
\]
by assumption;
\begin{align*}
  (a-1)[a^{k}+2(a^{k-1}+\cdots+a)+1]-a^k+a
  &= (a-1)(a^k+a^{k-1}+\cdots a+1)-a+a \\
  &= (a-1)(a^k+a^{k-1}+\cdots+a+1)
\end{align*}
and by the Second Principle of Finite Induction, 
$S=\mathbb{N}$.\hfill$\blacksquare$

\hfill\break
\textbf{4. }Prove that the cube of any integer can be written as the difference
of two squares. Notice that
\[
  n^3 = (1^3+2^3+\cdots+n^3)-(1^3+2^3+\cdots+(n-1)^3).
\]
$\pmb{\sharp4}.$ By the result of $\pmb{\sharp1(e)}$, for all $n\in\mathbb{N}$;
\[
  1^3+2^3+3^3+\cdots+n^3 = \left[\frac{n(n+1)}{2}\right]^2
\]
and since;
\begin{align*}
  n^3 &= (1^3+2^3+\cdots+n^3)-(1^3+2^3+\cdots+(n-1)^3)\\
      &=\left[\frac{n(n+1)}{2}\right]^2-\left[\frac{(n-1)(n-2)}{2}\right]^2
\end{align*}
the cube of any integer $n$ can be written as the difference of two 
squares.\hfill$\blacksquare$

\hfill\break
\textbf{5. }
\begin{enumerate}[label=(\alph*)]
  \item Find the values of $n\leq7$ for which $n!+1$ is a perfect square.
  \item True or false? For positive integers $m$ and $n$, $(mn)!=m!n!$ and 
    $(m+n)!=m!+n!$.
\end{enumerate}

\hfill\break
$\pmb{\sharp5.}$
\begin{enumerate}[label=(\alph*)]
  \item $n=4,5,7$.
  \item False by counter examples such as $m=2,n=3$ where
    \[
      (mn)!=6!=6\cdot5\cdot\cdots\cdot1=720\neq
      12=2\cdot3\cdot2\cdot1=m!n!
    \]
\end{enumerate}\hfill$\blacksquare$
\section{The Binomial Theorem}
\end{document}
