\documentclass[atbegshi, chapter]{memoir}
\usepackage{fapapersize}
\usefapapersize{*,*,30mm,*,30mm,*}
\usepackage{amsmath,amssymb,amsfonts,amsthm}
\usepackage{hyperref,enumitem,mdframed,euler}
\hypersetup{colorlinks=true, linkcolor=magenta,}
\chapterstyle{chappell}
\setlist{nosep}
\newtheorem{thm}{Theorem}[chapter]
\newtheorem{lem}[thm]{Lemma}
\newtheorem{exm}[thm]{Example}
\newtheorem*{cor}{Corollary}
\newtheorem*{WOP}{Well-Ordering Principle}
\title{Elementary Number Theory}
\author{KTW}
\date{\today}
\begin{document}
\maketitle\thispagestyle{empty}\newpage
\tableofcontents
\chapter{Preliminaries}
\section{Mathematical Induction}
\begin{WOP}
  Every nonempty set $S$ of nonnegative integers contains a least element;
  that is, there is some integer $a$ in $S$ such that $a\leq b$ for all 
  $b$'s belonging to $S$.
  \[
    \forall S \subseteq\mathbb{Z}_{>0}, \exists a\in S:\forall b\in S, a \leq b
  \]
\end{WOP}
\begin{thm}[Archimedian Property]
  If $a$ and $b$ are any positive integers, then there exists a positive
  integer $n$ such that $na\geq b$.
  \[
    \forall a,b\in\mathbb{Z}_{>0}, \exists n\in\mathbb{Z}_{>0}:na\geq b
  \]
\end{thm}
\begin{proof}
  For reductio ad aubsurdum, suppose that the statement of the therorem is
  false. Then:
  \begin{align*}
  &\neg(\forall a,b\in\mathbb{Z}_{>0},\exists n\in\mathbb{Z}_{>0}:na\geq b)\\
    \Leftrightarrow\;&\exists a,b\in\mathbb{Z}_{>0}: \forall n\in\mathbb{Z}_{>0}, na<b\\
  \Leftrightarrow\;&\exists a,b\in\mathbb{Z}_{>0}: \forall n\in\mathbb{Z}_{>0},0<b-na\\
  \Leftrightarrow\;&S=\{b-na\;|\;n\in\mathbb{Z}_{>0}\} = \mathbb{Z}_{>0}\\
  \Rightarrow\;&\exists b-ma\in S:\forall b-na\in S, b-ma\leq b-na
  \quad[\textrm{by the Well-Ordering Principle}]\\
  \Rightarrow\;& \exists b-ma, b-(m+1)a\in S:\forall b-na\in S,
  b-(m+1)a < b-ma\leq b-na \longrightarrow\bot \\
  \Rightarrow\;&\forall a,b\in\mathbb{Z}_{>0},\exists n\in\mathbb{Z}_{>0}:na\geq b
  \end{align*}
\end{proof}
\begin{thm}[First Principle of Finite Induction]
  Let $S$ be a set of positive integers with the following properties:
  \begin{enumerate}[label=(\alph*)]
    \item $\exists 1\in S$
    \item $\forall k\in S, \exists k+1\in S$
  \end{enumerate}
  Then $S$ is the set of all positive integers.
\end{thm}
\begin{proof}
  Let $S$ be a set of positive integers such that:
  \[
    \exists 1\in S\;\&\;\forall k\in S, \exists k+1\in S
  \]
  For reductio ad absurdum, 
  let $T$ be a nonempty set of all positive integers not in $S$, that is:
  \[
    T=\{t\in\mathbb{Z}_{>0}\;|\;t\notin S\}
  \]
  By the Well-Ordering Principle, $T$ has a least element $a$, and:
  \begin{align*}
    1\in S &\Rightarrow 1<a\in T \\
           &\Leftrightarrow 0<a-1\notin T \\
           &\Leftrightarrow a-1\in S\\
           &\Leftrightarrow a\in S\longrightarrow\bot\\
           &\Rightarrow T=\varnothing\\
           &\Leftrightarrow S = \mathbb{Z}_{>0}
  \end{align*}
\end{proof}
\begin{thm}[Second Principle of Finite Induction]
  Let $S$ be a set of positive integers with the following properties:
  \begin{enumerate}[label=(\alph*)]
    \item $\exists1\in S$
    \item $\exists 1,2,\ldots, k\in S\Rightarrow \exists k+1\in S$
  \end{enumerate}
  Then $S$ is the set of all positive integers.
\end{thm}
\begin{proof}
  Let $S$ be a set of positive integers follwing properties above.
  For reductio ad absurdum, let $T$ be a nonempty set of all positive integers
  not in $S$. By the Well-Ordering Principle, $T$ has a least element $a$, and:
  \begin{align*}
    1\in S &\Rightarrow 1<a\in T \\
           &\Leftrightarrow 0<1,\ldots,a-1\notin T\\
           &\Leftrightarrow 1,\ldots,a-1\in S \\
           &\Leftrightarrow a\in S \longrightarrow\bot\\
           &\Rightarrow T = \varnothing \\
           &\Leftrightarrow S = \mathbb{Z}_{>0}
  \end{align*}
\end{proof}
\begin{exm}[Lucas sequence]\normalfont
  \[
    1,3,4,7,11,18,29,47,76,\ldots
  \]
  Sequence above may be defined inductively by
  \[
    \begin{cases}
      a_1 = 1 \\
      a_2 = 3 \\
      a_n = a_{n-1}+a_{n-2}\quad\textrm{ for all }n\geq 3
    \end{cases}
  \]
  We contend that the inequality
  \[
    a_n < \left(\frac{7}{4}\right)^n
  \]
  holds for every positive integer $n$. First of all, for $n=1$ and $2$, we
  have
  \[
    a_1 = 1 < \left(\frac{7}{4}\right)^1\quad\&\quad
    a_2 = 3 < \left(\frac{7}{4}\right)^2 = 3\frac{1}{16}
  \]
  and this provides a basis for the induction. For the induction step,
  choose an integer $k\geq3$ and assume that the inequality is valid
  for $n=1,2,\ldots,k-1$. Then, in particular:
  \[
    a_{k-1}<\left(\frac{7}{4}\right)^{k-1}\quad\&\quad
    a_{k-2}<\left(\frac{7}{4}\right)^{k-2}
  \]
  By the way in which the sequence is formed, it follows that:
  \begin{align*}
    a_k = a_{k-1}+a_{k-2}
    &< \left(\frac{7}{4}\right)^{k-1}+\left(\frac{7}{4}\right)^{k-2} \\
    &= \left(\frac{7}{4}\right)^{k-2}\left(\frac{7}{4}+1\right)\\
    &=\left(\frac{7}{4}\right)^{k-2}\left(\frac{11}{7}\right)\\
    &<\left(\frac{7}{4}\right)^{k-2}\left(\frac{7}{4}\right)^{2}
    =\left(\frac{7}{4}\right)^{k}
  \end{align*}
  Because the inequality is true for $n=k$ whenever it is true for the
  integers $1,2,\ldots,k-1$, we conclude by the second induction principle
  that $a_n<(7/4)^n$ for all $n\geq 1$.
\end{exm}
\end{document}
