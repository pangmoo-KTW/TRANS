\documentclass[a4paper,chapter,atbegshi]{oblivoir}
\usepackage{fapapersize}
\usefapapersize{210mm,297mm,30mm,*,30mm,32mm}
\usepackage{amsmath, amsfonts, graphicx, xcolor, caption, enumitem, mdframed}
\usepackage{ob-chapstyles, listings,}
\usepackage[longend,ruled,vlined]{algorithm2e}
\hypersetup{colorlinks=true, linkcolor=magenta, citecolor=magenta, urlcolor=cyan,}
\chapterstyle{chappell}
\setlist{nosep}
\definecolor{mGreen}{rgb}{0,0.6,0}
\definecolor{mGray}{rgb}{0.5,0.5,0.5}
\definecolor{mPurple}{rgb}{0.58,0,0.82}
\definecolor{backgroundColour}{rgb}{0.95,0.95,0.92}
\lstdefinestyle{C}{
    backgroundcolor=\color{backgroundColour},   
    commentstyle=\color{mGreen},
    keywordstyle=\color{magenta},
    numberstyle=\tiny\color{mGray},
    stringstyle=\color{mPurple},
    basicstyle=\normalsize\ttfamily,
    breakatwhitespace=false,         
    breaklines=true,                 
    captionpos=b,                    
    keepspaces=true,                 
    numbers=left,                    
    numbersep=5pt,                  
    showspaces=false,                
    showstringspaces=false,
    showtabs=false,                  
    tabsize=2,
    language=C
}
\title{자료구조}
\author{김태원}
\date{\today}
\begin{document}
\maketitle
\chapter{\texttt{C} 리뷰}
\begin{mdframed}\textbf{1. }
  입력으로 하나의 양의 정수를 받은 후 $0$이 될 때까지 연속적으로 $2$로 나눈 몫을
  출력하는 프로그램을 작성하라.
\end{mdframed}
\begin{lstlisting}[style=C]
#include <stdio.h>
int main(){
  int n;
  scanf("%d", &n);
  for(int i=n; i!=0; i/=2)
    printf("%d ", i);
}
\end{lstlisting}
쉬운 문제다. \colorbox{yellow}{\ttfamily i/=2}는 그냥 멋내는 용이다.
\hfill\break
\begin{mdframed}\textbf{2. }
  입력으로 하나의 양의 정수 $n$을 받은 후 다음의 합을 구하여 출력하는 프로그램을
  작성하라. 단, 소수점 $4$자리까지만 출력하라.
  \[
    1 + \frac{1}{2}+\frac{1}{3}+\cdots+\frac{1}{n}
  \]
\end{mdframed}
\begin{lstlisting}[style=C]
#include <stdio.h>
int main(){
  int n;
  double sum = 0;
  for(double r=1; r<=n; r++)
    sum = sum + 1/r;
  printf("%.4f\n", sum);
}
\end{lstlisting}
5번 라인의 \texttt{for}문에서 \texttt{r}을 \colorbox{yellow}{\ttfamily double}로
선언했다. \texttt{int}로 선언하면 모든 입력에 대해 출력은 1이기 때문이다.
\hfill\break
\begin{mdframed}\textbf{3. }
  입력으로 하나의 양의 정수 $n$을 받은 후 다음의 합을 구하여 출력하는 프로그램을
  작성하라. 단, 소수점 4자리까지만 출력하라.
  \[
    1+\frac{1}{2!}+\frac{1}{3!}+\cdots+\frac{1}{n!}
  \]
\end{mdframed}
\begin{lstlisting}[style=C]
#include <stdio.h>
int main(){
  int n;
  double sum = 0;
  scanf("%d", &n);
  for(double i=1; i<=n; i++){
    double frac = 1;
    for(double j=i; j>0; j--)
      frac *= j;
    sum += 1/frac;
  }
  printf("%.4f\n", sum);
}
\end{lstlisting}
순서대로 해결했다. 9번 라인이 아니라 7번 라인에서 \texttt{frac}을 선언한
이유는 10번 라인에서 \texttt{frac}이 쓰이기 때문이다.
\hfill\break
\begin{mdframed}\textbf{4. }
  먼저 입력될 정수의 개수 $n\leq100$을 입력받고, 이어서 $n$개의 정수를 받아
  평균과 표준편차를 계산하여 소수점 이하 4번째 자리까지 출력하는 프로그램을
  작성하라. 표준편차는 다음과 같이 정의된다. 루트를 계산하기 위해서 
  \texttt{math.h}를 \texttt{include}하고 \texttt{sqrt}함수를 사용하라.
  \[
    SD = \sqrt{\frac{1}{N}\sum_{i=1}^N(x_i-\bar{x})^2}
  \]
\end{mdframed}
\begin{lstlisting}[style=C]
#include <stdio.h>
#include <stdlib.h>
#include <math.h>
int main(){
  int n;
  int *arr;
  double avg = 0;
  double sum = 0;
  scanf("%d", &n);
  arr = (int *)malloc(sizeof(int)*n);
  for(int i=0; i<n; i++){
    int p;
    scanf("%d", &p);
    arr[i] = p;
    avg += arr[i];
  }
  for(int i=0; i<n; i++)
    sum += pow(arr[i]-avg/n, 2.0);
  printf("%.5g\n", sqrt(sum/n));
  free(arr);
}
\end{lstlisting}
\colorbox{yellow}{\ttfamily .5g}라는 formant specifier를
사용했는데, 가령 $4.5000\ldots$를 $4.5$로 잘라서 출력하기 위해서다. 참고로
\texttt{math.h} 라이브러리를 포함한 파일을 컴파일하려면 \texttt{-lm} 명령어를
덧붙여야 한다.
\hfill\break
\begin{mdframed}\textbf{5. }
  먼저 입력될 정수의 개수 $2\leq n\leq 100$을 입력받고, 이어서 $n$개의 정수를
  입력받는다. 입력된 정수들 중에서 최소값과 두 번째로 작은 값을 찾아
  출력하는 프로그램을 작성하라. 만약 최소값이 2개 이상 중복되어 존재하면
  그 중 하나를 최소값으로, 다른 하나를 두 번째로 작은 값으로 간주한다.
\end{mdframed}
\begin{lstlisting}[style=C]
#include <stdio.h>
#include <stdlib.h>
int main(){
  int n, key;
  int *arr;
  scanf("%d", &n);
  arr = (int*)malloc(sizeof(int)*n);
  for(int i=0; i<n; i++)
    scanf("%d", &arr[i]);
  for(int i=1; i<n; i++){
    key = arr[i];
    int j;
    for(j=i-1; j>=0 && arr[j]>key; j--)
      arr[j+1] = arr[j];
    arr[j+1] = key;
  }
  printf("%d %d\n", arr[0], arr[1]);
  free(arr);
}
\end{lstlisting}
10번 라인에서 시작되는 \texttt{for} 블록은 삽입정렬 알고리즘을 구현한 것이다.
\begin{algorithm}
  \caption{Insertion Sort}
  \KwIn{$A$: Array}
  \For{$i\in\lbrace2,\ldots, A.length\rbrace$}{
    $key \gets A[i]$\\
    $j\gets i-1$ \\
    \While{$j>0\;\&\; A[j]>key$}{
      $A[j+1]\gets A[j]$ \\
      $j\gets j-1$ 
    }
    $A[j+1]\gets key$
  }
\end{algorithm}
삽입 정렬 알고리즘은 아래와 같은 방식으로 작동한다.
\begin{align*}
  A =
  [7,3,1,2,4,6]\quad &\textrm{ 첫 번째 for 루프: }7=A[1]>key=A[2]=3 \\ 
                &\mapsto [3,7,1,2,4,6]\\
                & \textrm{ 두 번째 for 루프: }7=A[2]>key=A[3]=1\\
                &\mapsto [3,1,7,2,4,6] \\
                & \textrm{ 두 번째 for 루프: }3=A[1]>key=A[3]=1 \\
                &\mapsto [1,3,7,2,4,6] \\
                &\textrm{ 세 번째 for 루프: }7=A[3]>key=A[4]=2\\
                &\mapsto [1,3,2,7,4,6] \\
                &\textrm{ 세 번째 for 루프: }3=A[2]>key=A[4]=2\\
                &\mapsto [1,2,3,7,4,6] \\
                &\textrm{ 네 번째 for 루프: }7=A[4]>key=A[5]=4\\
                &\mapsto [1,2,3,4,7,6]\\
                &\textrm{ 다섯 번째 for 루프: }7=A[5]>key=A[6]=6\\
                &\mapsto [1,2,3,4,6,7]
\end{align*}
\begin{mdframed}\textbf{6. }
  수열에서 큰 값이 작은 값보다 앞서 나오는 경우 두 값을 역전된 쌍이라고 부른다.
  예를 들어 수열 $4,2,1,1,3$에는 $(4,2),(4,1),(4,1),(4,3),(2,1),(2,1)$의 총
  6개의 역전된 쌍이 있다. 수열을 입력으로 받아서 역전된 쌍의 개수를 카운트하여
  출력하는 프로그램을 작성하라. 키보드로부터 먼저 정수의 개수 $N$을 입력받고,
  이어서 $N$개의 정수를 입력 받는다.
\end{mdframed}
\begin{lstlisting}[style=C]
#include <stdio.h>
#include <stdlib.h>
int main(){
  int n, gncount;
  int* arr;
  scanf("%d", &n);
  arr = (int*)malloc(sizeof(int)*n);
  for(int i=0; i<n; i++)
    scanf("%d", &arr[i]);
  for(int i=0; i<n; i++){
    int gn = arr[i];
    for(int j=i; j<n; j++){
      if(gn > arr[j])
        gncount++;
    }
  }
  free(arr);
  printf("%d\n", gncount);
}
\end{lstlisting}
\texttt{for}문을 돌면서 각 성분에 대해 그 성분보다 큰 것들을 세는 단순무식한
코드다.
\hfill\break
\begin{mdframed}\textbf{7. }
  키보드로부터 연속해서 음이 아닌 정수들을 입력받는다. 정수가 하나씩 입력될
  때마다 현재까지 입력된 정수들을 오름차순으로 정렬하여 화면에 출력한다. 단,
  새로 입력된 정수가 이미 이전에 입력된 정수라면 \texttt{duplicate}라고 출력하고
  저장하지 않고 버린다. 사용자가 $-1$을 입력하면 프로그램을 종료한다. 입력되는
  정수의 개수는 100개를 넘지 않는다.
\end{mdframed}
\begin{lstlisting}[style=C]
#include <stdio.h>
int check(int* arr, int size, int n);
void insertionSort(int* arr, int size);
int main(){
  int arr[100];
  scanf("%d", &arr[0]);
  printf("%d\n\n", arr[0]);
  for(int i=1; i<100; i++){
    int n;
    scanf("%d", &n);
    if(n==-1)
      break;
    if(check(arr, i, n)){
      arr[i] = n;
      insertionSort(arr, i);
      for(int j=0; j<=i; j++)
        printf("%d ", arr[j]);
    }
    else{
      i--;
      printf("duplicate");
    }
    printf("\n\n");
  }
}
int check(int* arr, int size, int n){
  int result;
  for(int i=0; i<size; i++){
    if(arr[i]==n){
      result = 0;
      break;
    }
    else
      result = 1;
  }
  return result;
}
void insertionSort(int* arr, int size){
  for(int i=1; i<=size; i++){
    int key = arr[i];
    int j;
    for(j=i-1; j>=0 && arr[j]>key; j--)
      arr[j+1] = arr[j];
    arr[j+1] = key
  }
}
\end{lstlisting}
\hfill\break
\begin{mdframed}\textbf{8. }
  먼저 입력될 정수의 개수 $n\leq 100$을 입력받고, 이어서 $n$개의 정수를 받아
  순서대로 배열에 저장한다. 그런 다음 키보드로부터 다시 하나의 정수 $k$를
  입력받은 후 $n$개의 정수들 중에서 $k$에 가장 가까운, 즉 $k$와의 차이의
  절대값이 가장 작은 정수를 찾아 출력하는 프로그램을 작성하라. 
\end{mdframed}
\begin{lstlisting}[style=C]
#include <stdio.h>
#include <stdlib.h>
int main(){
  int n, k, diff;
  int* arr;
  scanf("%d", &n);
  arr = (int*)malloc(sizeof(int)*n);
  for(int i=0; i<n; i++)
    scanf("%d", &arr[i]);
  scanf("%d", &k);
  diff = arr[0];
  for(int i=0; i<n; i++){
    if(abs(k-diff) > abs(k-arr[i]))
      diff = arr[i];
  }
  free(arr);
  printf("%d\n", diff);
}
\end{lstlisting}
\colorbox{yellow}{\ttfamily abs} 함수를 사용한 단순무식한 방법이다.
\hfill\break
\begin{mdframed}\textbf{9. }
  사용자로부터 $n<100$개의 정수를 입력받아 크기순으로 정렬한 후 중복된 수를
  제거하는 프로그램을 작성하라. 입력 형식은 먼저 $n$의 값이 주어지고 이어서
  $n$개의 정수들이 주어진다. 예를 들어 $n=8$이고 입력된 정수들이 $4,7,4,12,4
  10,9,7$이라면 중복을 제거하고 남은 정수들은 $4,7,9,10,12$이다. 그러면 먼저
  남은 정수의 개수 $5$를 출력하고 콜론을 출력한 후 남은 정수들을 오름차순으로
  출력한다.
\end{mdframed}
\begin{lstlisting}[style=C]
#include <stdio.h>
#include <stdlib.h>
void insertionSort(int* arr, int size);
int main(){
  int n;
  int count = 0;
  int index = 0;
  int *temp, *arr;
  scanf("%d", &n);
  temp = (int*)malloc(sizeof(int)*n); 
  arr = (int*)malloc(sizeof(int)*count);
  for(int i=0; i<n; i++)
    scanf("%d", &temp[i]);
  for(int i=0; i<n; i++){
    int check = 0;
    for(int j=0; j<i; j++){
      if(temp[i]==temp[j]){
        check = 1;
        break;
      }
    }
    if(check == 0){
      count++;
      arr[index] = temp[i];
      index++;
    }
  }
  free(temp);
  insertionSort(arr, count);
  printf("%d: ", count);
  for(int i=0; i<count; i++)
    printf("%d ", arr[i]);
  printf("\n");
  free(arr);
}
void insertionSort(int* arr, int size){
  for(int i=1; i<size; i++){
    int key = arr[i];
    int j;
    for(j=i-1; j>=0 && arr[j]>key; j--)
      arr[j+1] = arr[j];
    arr[j+1] = key;
  }
}
\end{lstlisting}
다양한 불변항을 사용했다.
\hfill\break
\begin{mdframed}\textbf{10. }
  정렬을 하는 가장 간단한 방법 중의 하나는 다음과 같이 하는 것이다. 배열
  \texttt{data}에 \texttt{data[0]}에서 \texttt{data[n-1]}까지 $n$개의 정수가
  저장되어 있다. 먼저 \texttt{data[0]}와 \texttt{data[n-1]} 사이의 정수들
  중에서 가장 큰 정수를 찾는다. 그것을 \texttt{data[k]}라고 가정해보자.
  그러면 \texttt{data[k]}와 \texttt{data[n-1]}을 교환한다. 이제 가장 큰
  정수가 \texttt{data[n-1]}, 즉 맨 마지막 위치에 저장되었으므로 그 값에 대해서는
  더 이상 생각할 필요가 없다. 이제 \texttt{data[0] - data[n-2]} 중에서 최대값을
  찾는다. 그 값을 \texttt{data[p]}라고 하자. 그러면 다시 \texttt{data[p]}와
  \texttt{data[n-2]}를 교환하고 \texttt{data[n-2]}에 대해서는 잊어버려도 된다.
  이런 식으로 계속하면 마지막에는 \texttt{data[0]}과 \texttt{data[1]} 중에
  최대값을 \texttt{data[1]}과 교환하면 전체의 정렬이 완료된다. 이 알고리즘을
  구현하라. 입력은 먼저 정렬할 개수 $n\leq 100$이 주어지고 이어서 $n$개의
  정수들이 주어진다.
\end{mdframed}
\begin{lstlisting}[style=C]
#include <stdio.h>
#include <stdlib.h>
void selectionSort(int* arr, int size);
void swap(int* a, int* b);
int main(){
  int n;
  int* data;
  scanf("%d", &n);
  data = (int*)malloc(sizeof(int)*n);
  for(int i=0; i<n; i++)
    scanf("%d", &data[i]);
  selectionSort(data, n);
  for(int i=0; i<n; i++)
    printf("%d ", data[i]);
  printf("\n");
  free(data);
}
void selectionSort(int* arr, int size){
  int i, j, max;
  for(i=size-1; i>0; i--){
    max = i;
    for(j=i-1; j>=0; j--)
      if(arr[j] > arr[max])
        max=j;
      swap(&arr[max], &arr[i]);
  }
}
void swap(int* a, int* b){
  int temp = *a;
  *a = *b;
  *b = temp;
}
\end{lstlisting}
선택정렬 알고리즘에 대한 문제다. 배열 성분 자체를 최대값으로 두지 않고
최대값의 성분에 대응하는 인덱스로 정렬한다는 점에 유의하라. 보통 선택
정렬은 최대값이 아니라 최소값을 기준으로 한다. 불변항만 살짝 바꾸면 된다.
\begin{algorithm}
  \caption{Selection Sort}
  \KwIn{$A$: Array}
  \For{$i\in\lbrace1,\ldots, A.length-1\rbrace$}{
    $min \gets i$\\
    \For{$j\in\lbrace i+1, A.length-1$}{
      \If{$A[j]<A[min]$}{
        $min\gets j$
      }
    }
  \If{$min\neq i$}{
    swap($A[i],A[min]$)
    }
  }
\end{algorithm}
\begin{mdframed}\textbf{11. }
  입력으로 하나의 문자열을 받은 후 뒤집어서 출력하는 프로그램을 작성하라. 예를
  들어 \texttt{hello}를 입력하면 \texttt{olleh}가 출력된다.
\end{mdframed}
\begin{lstlisting}[style=C]
#include <stdio.h>
#include <string.h>
void revstr(char *str);
int main(){
  char* str;
  scanf("%s", str);
  revstr(str);
  printf("%s\n", str);
  return 0;
}
void revstr(char *str){
  int len = strlen(str);
  for(int i=0; i<len/2; i++){
    int temp = str[i];
    str[i] = str[len-i-1];
    str[len-i-1] = temp;
  }
}
\end{lstlisting}
단순하게 swap하여 해결했다.
\hfill\break
\begin{mdframed}\textbf{12. }
  영문 소문자로 구성된 하나의 문자열을 입력받은 후 문자열을 구성하는 문자들을
  알파벳 순으로 정렬하여 만들어지는 문자열을 출력하라. 예를 들어 \texttt{hello}가
  입력되면 \texttt{ehllo}를 출력한다.
\end{mdframed}
\begin{lstlisting}[style=C]
#include <stdio.h>
#include <string.h>
void alphabetOrder();
int main(){
  char *ch;
  scanf("%s", ch);
  alphabetOrder(ch);
  puts(ch);
}
void alphabetOrder(char *ch){
  char temp;
  int i, j, length = strlen(ch);
  for(i=0; i<length; i++){
    for(j=i+1; j<length; j++){
      if(ch[i] > ch[j]){
        temp = ch[i];
        ch[i] = ch[j];
        ch[j] = temp;
      }
    }
  }
}
\end{lstlisting}
기본적으로 선택정렬 알고리즘의 응용이다.
\hfill\break
\begin{mdframed}\textbf{13. }
  아나그램이란 문자들의 순서를 재배열하여 동일하게 만들 수 있는 문자열을
  말한다. 대소문자는 구분하지 않는다. 예를 들어서 \texttt{Silent}와
  \texttt{Listen}은 아나그램이다. 입력으로 두 문자열을 받아서 아나그램인지
  판단하는 프로그램을 작성하라.
\end{mdframed}
\begin{lstlisting}[style=C]
#include <stdio.h>
#include <string.h>
void cnvt_lwr(char *str);
int anagram(char *str1, char *str2);
int main(){
  char *words[2];
  for(int i=0; i<2; i++){
    char buf[100];
    scanf("%s", buf);
    words[i] = strdup(buf);
    cnvt_lwr(words[i]);
  }
  if(anagram(words[0], words[1]))
    puts("yes");
  else
    puts("no");
}
void cnvt_lwr(char *str){
  for(int i=0; i<strlen(str); i++){
    if(str[i] >= 65 && str[i] <= 90)
      str[i] = str[i]+32;
  }
}
int anagram(char *str1, char *str2){
  if(strlen(str1)!=strlen(str2))
    return 0;
  int count = 0;
  for(int i=0; i<strlen(str1); i++){
    for(int j=0; j<strlen(str1); j++){
      if(str1[i]==str2[j]){
        count++;
        break;
      }
    }
  }
  if(count==check)
    return 1;
  else
    return 0;
}
\end{lstlisting}
\texttt{main}에서 중요한 것은 \colorbox{yellow}{\texttt{buf}와 \texttt{strdup}을
이용한 입력}이다. \texttt{anagram}은 단순무식한 논리를 \texttt{break}로
구현한 것이고 \texttt{cnvt\_lwr}는 ASCII 값을 활용하는 함수이므로
그냥 외워두는 편이 나을 것 같다.
\begin{mdframed}\textbf{14. }
  영문 소문자로 구성된 2개의 단어를 입력받은 후 두 단어가 동일한 문자집합으로
  구성되었는지 검사하여 \texttt{yes} 혹은 \texttt{no}를 출력하는 프로그램을
  작성하라. 예를 들어 \texttt{ababc}와 \texttt{cba}는 문자집합 
  \texttt{\{a,b,c\}}로 구성되었으므로 \texttt{yes}다. 입력 단어의 길이는 20이하다.
\end{mdframed}
\begin{lstlisting}[style=C]
#include <stdio.h>
#include <string.h>
int alphSet(int* wordA, int* wordB);
int main(){
  char* words[2];
  char alphabet[26];
  int firstAlphCount[26], secAlphCount[26];
  for(int i=0; i<2; i++){
    char buf[20];
    scanf("%s", buf);
    words[i] = strdup(buf);
  }
  for(int i=0; i<26; i++){
    alphabet[i] = 'a'+i;
    firstAlphCount[i] = 0;
    secAlphCount[i] = 0;
  }
  for(int i=0;i<strlen(words[0]);i++){
    for(int j=0;j<26;j++){
      if(words[0][i] == alphabet[j])
        firstAlphCount[i] = 1;
    }
  }
  for(int i=0;i<strlen(words[1]);i++){
    for(int j=0;j<26;j++){
      if(words[1][i] == alphabet[j])
        secAlphCount[i] = 1;
    }
  }
  if(alphSet(firstAlphCount, secAlphCount))
    puts("yes");
  else
    puts("no");
}
int alphSet(int* wordA, int* wordB){
  for(int i=0; i<26; i++){
    if(wordsA[i] != wordsB[i])
      return 0;
  }
  return 1;
}
\end{lstlisting}
단순무식하지만 뭘 하려는지 잘 보인다. 아무튼 14번 라인인
\colorbox{yellow}{\texttt{alphabet[i] = 'a'+i}}가 핵심이다.
\hfill\break
\begin{mdframed}\textbf{15. }
  입력으로 $n<100$개의 영문 문자열을 받는다. 각 문자열의 길이는 20이하다. 이
  문자열들을 문자열의 길이가 짧은 것부터 긴 것 순서로 정렬하여 출력하라. 단,
  길이가 동일한 문자열들은 그들끼리 사전식 순서로 정렬해야 한다. 입력 형식은
  먼저 문자열의 개수 $n$이 주어지고, 이어서 $n$개의 문자열이 한 줄에 하나씩
  주어진다. 
\end{mdframed}
\begin{lstlisting}[style=C]
#include <stdio.h>
#include <stdlib.h>
#include <string.h>
int main(){
  int n;
  char **words;
  scanf("%d", &n);
  words = (char **)malloc(sizeof(char[101])*n);
  for(int i=0; i<n; i++){
    char buf[101];
    scanf("%s", buf);
    words[i] = strdup(buf);
  }
  for(int i=0; i<n; i++){
    for(int j=0; j<n; j++){
      if(strlen(words[j])>strlen(words[i])){
        char* temp1;
        temp1 = words[i];
        words[i] = words[j];
        words[j] = temp1;
      }
      else if(strlen(words[j])==strlen(words[i])){
        if(words[j][0] > words[i][0]){
          char* temp2;
          temp2 = words[i];
          words[i] = words[j];
          words[j] = temp2;
        }
      }
    }
  }
  printf("\n");
  for(int i=0; i<n; i++)
    printf("%s\n", words[i]);
  free(words);
}
\end{lstlisting}
8번 라인의 동적 할당에서 쓰인 이중 포인터를 이해하는 것이 관건이다. 다시 말해
14번 라인 이하의 \texttt{for} 블록에서 swap되는 \texttt{words[]}는 문자가 아니라
문자열 자체다.
\hfill\break
\begin{mdframed}\textbf{16. }
  입력으로 텍스트 파일 \texttt{harry.txt}를 읽어서 이 파일에 등장하는 모든
  단어의 목록을 중복된 단어 없이 사전식 순서로 정렬한다. 이제 새로운 단어
  하나를 키보드로부터 입력 받는다. 저장된 단어들 중에서 이 새로운 단어를
  접두어(prefix)로 하는 모든 단어를 찾아서 한 줄에 하나씩 출력하는 프로그램을
  작성하라. 마지막으로 찾아진 단어의 개수를 출력하라. 어떤 단어가 다른 단어의
  접두어인지는 표준 라이브러리가 제공하는 \texttt{strncmp} 함수를 이용하여 검사할
  수 있다. 이 함수의 사용 방법은 검색하여 알아보라.
\end{mdframed}
\begin{lstlisting}[style=C]
#include <stdio.h>
#include <string.h>
int prefix(char *pre, char *words);
int main(){
  char *words[100000];
  char buf[100];
  int n = 0;
  FILE *fp = fopen("harry.txt", "r");
  while(fscanf(fp, "%s", buf)!=EOF){
    int i = 0;
    for(;i<n;i++){
      if(strcmp(buf,words[i])==0)
        break;
    }
    if(i==n)
      words[n++] = strdup(buf);
  }
  fclose(fp);
  for(int i=0; words[i]; i++){
    for(int j=0; words[j]; j++){
      if(strcmp(words[i], words[j])<0){
        char *tmp = words[i];
        words[i] = words[j];
        words[j] = tmp;
      }
    }
  }
  scanf("%s", buf);
  printf("\n");
  int count = 0;
  for(int i=0; words[i]; i++){
    if(prefix(buf, words[i])){
      printf("%s\n", words[i]);
      count++;
    }
  }
  printf("%d\n", count);
  return 0;
}
int prefix(char *pre, char *words){
  return strncmp(pre, words, strlen(pre))==0;
}
\end{lstlisting}
코드가 꽤 복잡하다. 우선 
\begin{lstlisting}[style=C]
char *words[100000];
\end{lstlisting}
은 문자열(\texttt{char *}) 100000개로 구성된 배열을 만든다. 
\begin{lstlisting}[style=C]
while(fscanf(fp, "%s", buf)!=EOF)
\end{lstlisting}
는 입력된 파일이 파일의 끝(\texttt{EOF})에 이를 때까지 문자열을 한 줄 씩 읽는다.
이 \texttt{while}문 전체를 보자.
\begin{lstlisting}[style=C]
while(fscanf(fp, "%s", buf)!=EOF){
  int i = 0;
  for(; i<n; i++){
    if(strcmp(buf, words[i])==0)
      break;
  }
  if(i==n)
    words[n++] = strdup(buf);
}
\end{lstlisting}
앞서 선언한 문자열 배열 \texttt{words}의 인덱스는 \texttt{n}으로, 버퍼에서
한 번씩 복제(\texttt{strdup})할 때마다 \texttt{n}이 증가한다. 
또한 문자열을 한 줄씩 읽는 \texttt{while}문의 인덱스는 \texttt{i}인데,
\texttt{while}문 내부의 \texttt{for}문도 \texttt{i}를 사용한다.
문자열 배열상의 문자열 개수 \texttt{n}번 동안 버퍼와 \texttt{i}번째
문자열 배열을 비교(\texttt{strcmp})하여 일치하면 \texttt{for}문에서
\texttt{break}한다. 이때 \texttt{i}는 \texttt{for}문 외부에서
선언되었기에 증가된 \texttt{i} 혹은 중복이 발생한 \texttt{i}의 값이 유지된다.
이때 \texttt{i}와 \texttt{n}이 일치하면 문자열 배열에 버퍼의 문자열을
복제하고 \texttt{n}을 증가하는 것이다. 좀 꼬였는데, 중복 없이
입력하기 위한 것이다.

\texttt{prefix} 함수에 사용된 \texttt{strncmp}는 길이를 지정하여
\texttt{strcmp}하는 함수다.
\hfill\break
\begin{mdframed}\textbf{17. }
  입력으로 텍스트 파일 \texttt{harry.txt}를 읽는다. 이 텍스트 파일은 오직
  영문 소문자만으로 구성되어 있다. 이 파일에 등장하는 길이가 6이상인 모든
  단어의 목록과 각 단어의 등장 빈도를 구하여 화면으로 출력하는 프로그램을
  작성하라. 단어들은 사전식 순서로 정렬되어 출력되어야 한다. 출력 파일의
  각 줄에 하나의 단어와 그 단어의 등장 빈도를 콜론으로 구분하여 출력하라.
  동일한 단어가 중복해서 출력되어서는 안 된다. 출력의 마지막에는 전체 단어의
  개수를 출력하라. 아래는 올바른 출력의 시작 부분과 끝 부분을 보여준다.
\end{mdframed}
\begin{lstlisting}[style=C]
#include <stdio.h>
#include <string.h>
int wordCount(char* file, char* word);
int main(){
  char *words[100000];
  int count, n = 0;
  FILE *fp = fopen("harry.txt", "r");
  char buf[100];
  while(fscanf(fp, "%s", buf)!=EOF){
    if(strlen(buf)>=6){
      int i = 0;
      for(;i<n;i++){
        if(strcmp(buf,words[i])==0)
          break;
      }
      if(i==n)
        words[n++] = strdup(buf);
    }
  }
  fclose(fp);
  for(int i=0; words[i]; i++){
    for(int j=0; words[j]; j++){
      if(strcmp(words[i],words[j])<0){
        char *temp = words[i];
        words[i] = words[j];
        words[j] = temp;
      }
    }
  }
  for(count=0; count<n; count++)
    printf("%s: %d\n", words[count], wordCount("harry.txt",words[count]));
  printf("%d\n", count);
}
int wordCount(char *file, char *word){
  int wc = 0;
  FILE *txt = fopen(file, "r");
  char buf[100];
  while((fgets(buf, 100, txt))!=NULL){
    if((strstr(buf, word))!=NULL)
      wc++;
  }
  fclose(txt);
  return wc;
}
\end{lstlisting}
\texttt{while}문은 앞선 문제와 일치한다. \texttt{wordCount}에서 쓴
\texttt{strstr} 함수는 \texttt{file}에 \texttt{word}가 나타나지 않으면 
\texttt{NULL}을 반환한다. 즉 \texttt{file}에 \texttt{word}가 이미 있으면
\texttt{wc}를 하나 증가시키는 함수가 \texttt{wordCount}다.
\hfill\break
\begin{mdframed}\textbf{19. }
  하나의 영문 소문자로 된 문자열이 입력으로 주어진다. 이 문자열에서 자음이
  가장 여러 번 연속해서 등장하는 부분을 찾아서 그 부분을 출력하는 프로그램을
  작성하라. 예를 들어 문자열 ``nietzsche''에서는 ``tzsch''가 가장 긴 연속된
  자음이다. 입력 문자열의 길이는 100이하이고, `a', `e', `i', `o', `u'를 제외한
  모든 알파벳은 자음으로 간주한다. 
\end{mdframed}
\begin{lstlisting}[style=C]
#include <stdio.h>
#include <string.h>
int isVowel(char c);
int main(){
  char input[101];
  scafn("%s", input);
  int maxConVow = 0;
  int curConVow = 0;
  int index = 0;
  for(int i=0; i<strlen(input); i++){
    if(isVowel(input[i]))
      curConVow++;
    else{
      if(curConVow > maxConVow){
        maxConVow = curConVow;
        index = i- maxConVow;
      }
      curConVow = 0;
    }
  }
  if(curConVow > maxConVow){
    maxConVow = curConVow;
    index = length - maxConVow;
  }
  if(maxConVow > 0){
    for(int i=index; i<index+maxConVow; i++)
      printf("%c", input[i]);
    printf("\n");
  }
}
int isVowel(char c){
  return(c != 'a' && c!= 'e' && c != 'i' && c != 'o' && c != 'u');
}
\end{lstlisting}
\chapter{문자열}
\begin{mdframed}\textbf{1. }
  하나의 구간(interval)은 시작점과 끝점의 좌표로 정의된다. 구간의 시작점과 끝점은
  정수이고, 끝점은 항상 시작점보다 크거나 같다. 먼저 구간의 개수 $n<100$이
  주어지고, 이어서 $n$개의 구간이 입력으로 주어진다. 그런 다음 다시 추가로 하나의
  구간이 주어진다. 추가로 주어진 구간에 완전히 포함되면서 가장 긴 구간을 찾아 그
  길이를 출력하는 프로그램을 작성하라.
\end{mdframed}
\begin{lstlisting}[style=C]
#include <stdio.h>
#include <stdlib.h>
int eval(int (*arr)[2], int (*arr)[2], int size);
int main(){
  int n;
  scanf("%d", &n);
  int (*interval)[2] = malloc(sizeof(int[n][2]));
  for(int i=0; i<n; i++){
    for(int j=0; j<2; j++)
      scanf("%d", &interval[i][j]);
  }
  int newval[1][2];
  for(int i=0; i<2; i++)
    scanf("%d", &newval[0][i]);
  printf("%d\n", eval(interval, newval, n));
  free(interval);
}
int eval(int (*arr1)[2], int (*arr2)[2], int size){
  int j=0;
  int length[100];
  for(int i=0; i<size; i++){
    if(arr2[0][0] <= arr1[i][0] && arr2[0][1] >= arr1[i][1]){
      length[j] = arr1[i][1]-arr1[i][0];
      j++;
    }
  }
  int maxLength = 0;
  for(int i=0; i<j; i++){
    if(maxLength <= length[i])
      maxLength = length[i];
  }
  return maxLength;
}
\end{lstlisting}
우선 \texttt{int (*arr)[2]}은 정수형 포인터 배열을 나타낸다. 
\texttt{int (*interval)[2]}를 선언한 다음, \texttt{int[n][2]}의 크기로
\texttt{malloc}한다. 즉 2차원 배열이다.
\hfill\break
\begin{mdframed}\textbf{2. }
  입력으로 $n<100$개의 구간이 주어진다. 각 구간은 구간의 시작점과 끝점으로
  표현된다. 각 구간의 시작점과 끝점은 정수이고, 끝점은 항상 시작점보다 크거나
  같다. 이 구간들을 시작점이 빠른 순서대로 정렬하여 출력하는 프로그램을
  작성하라. 시작점이 같은 경우 끝점이 빠른 것을 먼저 출력한다. 입력 형식은 
  먼저 $n$의 값이 주어지고, 이어서 각 구간의 시작점과 끝점이 차례대로 주어진다.
\end{mdframed}
\begin{lstlisting}[style=C]
#include <stdio.h>
#include <stdlib.h>
void psuedoInsertionSort(int *arr1, int *arr2, int size);
int main(){
  int n;
  scanf("%d", &n);
  int *first = (int*)malloc(sizeof(int)*n);
  int *second = (int*)malloc(sizeof(int)*n);
  for(int i=0; i<n; i++){
    int num1, num2;
    scanf("%d", &num1);
    scanf("%d", &num2);
    first[i] = num1;
    second[i] = num2;
  }
  psuedoInsertionSort(first, second, n);
  printf("\n");
  for(int i=0; i<n; i++)
    printf("%d %d\n", first[i], second[i]);
  free(first);
  free(second);
}
void psuedoInsertionSort(int *arr1, int *arr2, int size){
  int i, j, key1, key2;
  for(i=1; i<size; i++){
    key1 = arr1[i];
    key2 = arr2[i];
    for(j=i-1; j>=0 && (arr1[j]>key1 || (arr1[j]==key1&&arr2[j]>key2)); j--){
      arr1[j+1] = arr1[j];
      arr2[j+1] = arr2[j];
    }
    arr1[j+1] = key1;
    arr2[j+1] = key2;
  }
}
\end{lstlisting}
앞선 문제와 다르게 이차원 배열을 그냥 배열 두 개로 구현할 수도 있다.
\hfill\break
\begin{mdframed}\textbf{3. }
  키보드로부터 여러 개의 단어로 이루어진 한 라인의 텍스트를 입력받은 후
  단어와 단어 사이에 있는 하나의 공백 문자를 제외한 모든 공백 문자를 제거하고
  출력하는 프로그램을 작성하라. 또한 마지막에 출력된 문자열의 길이를 출력한다.
\end{mdframed}
\begin{lstlisting}[style=C]
#include <stdio.h>
#include <ctype.h>
#include <string.h>
#include <stdlib.h>

int main(){
  char in[1000];
  fgets(in, sizeof(in), stdin);
  char *out;
  out = (char*)malloc(sizeof(char)*length);
  int index = 0;
  int space = 1;
  for(int i=0; i<strlen(in); i++){
    if(isspace(in[i])){
      if(!space){
        out[index++] = ' ';
        space = 1;
      }
    }else{
      out[index++] = in[i];
      space = 0;
    }
  }
  if(index > 0 && out[index-1] == ' ')
    index--;
  out[index] = '\0';
  printf("%s:%d\n", out, strlen(out));
  free(out);
}
\end{lstlisting}
\texttt{scanf}가 아니라 \texttt{fgets}를 사용하는 이유는 공백 문자를 포함하여
입력하기 위한 것이다. 또한 \texttt{isspace} 함수는 \texttt{ctype.h} 헤더에
포함되어 있으며, 어떤 문자가 공백 문자인지 판별한다.
\hfill\break
\begin{mdframed}\textbf{4. }
  사전 파일 \texttt{shuffled\_dict.txt}을 읽는다. 이 파일에는 각 줄마다 하나의
  ``단어''그 단어에 대한 ``설명''이 저장되어 있다. ``단어''와 그 단어에 대한
  ``설명''은 하나의 탭 문자로 구분되어 있다. ``단어''는 하나의 영문 소문자
  문자열이며, ``설명''은 여러 개의 단어로 구성된 문장이다. 이 사전 파일에서
  단어들은 사전식 순서로 정렬되어 있지 않다. 이 파일을 읽은 후 단어들을 사전식
  순서로 정렬하여 ``\texttt{sorted\_dict.txt}''라는 이름의 새로운 파일을 생성하는
  프로그램을 작성하라. 저장된 파일에서는 한 줄에 하나의 단어와 그 단어에 대한
  설명을 탭 문자로 구분하여 저장해야 한다.
\end{mdframed}
\begin{lstlisting}[style=C]
#include <stdio.h>
#include <string.h>
int main(){
  char *words[100000];
  char buf[1000];
  int n = 0;
  FILE *txt = fopen("shuffled_dict.txt", "r");
  FILE *ntxt = fopen("sorted_dict.txt", "w");
  while(fgets(buf, 1000, txt)!=NULL)
    words[n++] = strdup(buf);
  for(int i=0; words[i]; i++){
    for(int j=0; words[j]; j++){
      if(strcmp(words[i],words[j])<0){
        char *tmp = words[i];
        words[i] = words[j];
        words[j] = tmp;
      }
    }
  }
  fclose(txt);
  for(int i=0; i<n; i++){
    char *r = strchr(words[i], '\t');
    *r = '\t';
    fprintf(ntxt, "%s\n", words[i]);
  }
  fclose(ntxt);
}
\end{lstlisting}
\texttt{strchr} 함수는 문자열에서 특정 문자를 찾은 다음 그 주소를 리턴한다.
사실 \texttt{shuffled\_dict.txt} 파일 자체가 탭 문자로 구분되어 있기에
정렬 말고는 할 일이 없다.
\hfill\break
\begin{mdframed}\textbf{5. }
  \texttt{table.txt}를 읽어서 \texttt{output.txt} 파일에 테이블 형태로
  출력하는 프로그램을 작성하라. 입력 파일의 첫 줄에는 테이블 행의 개수
  $m\leq10$과 열의 개수 $n\leq10$이 주어진다. 이어진 $m$줄에는 각 줄마다
  테이블의 한 행에 들어갈 $n$개의 내용이 문자 \texttt{\&}로 구분되어
  주어진다. 테이블의 어떤 칸은 빈 칸일 수도 있다는 것을 유념하라. 빈 칸의
  경우에도 하나 이상의 공백 문자로 표현되어 있다. 불필요한 공백 문자들을
  제거하여 최대한 깔끔하게 출력되도록 하라.
\end{mdframed}
\begin{lstlisting}[style=C]
#include <stdio.h>
#include <stdlib.h>
#include <string.h>

#define MAX_ROWS 100
#define MAX_COLS 100

int main(){
  int rows, cols;
  int width[MAX_COLS] = {0};
  char buf[256];
  char table[MAX_ROWS][MAX_COLS][256];
  FILE *input, *output;
  input = fopen("table.txt", "r");
  output = fopen("output.txt", "w");
  if(fgets(buf, sizeof(buf), input)!=NULL)
    sscanf(buf, "%d %d", &rows, &cols);
  for(int row=0; row<rows; row++){
    if(fgets(buf, sizeof(buf), input)==NULL)
      break;
    char *token = strtok(buf, "&");
    int col = 0;
    while(token != NULL && col < cols){
      strcpy(table[row][col], token);
      int tokLen = strlen(table[row][col]);
      if(tokLen > width[col])
        width[col] = tokLen;
      token = strtok(NULL, "&");
      col++;
    }
  }
  for(int row=0; row<rows; row++){
    for(int col=0; col<cols; col++){
      fprintf(output, "%-*s", width[col], table[row][col]);
      if(col <cols-1)
        fprintf(output, " ");
    }
    fprintf(output, "\n");
  }
  fclose(input);
  fclose(output);
}
\end{lstlisting}
\texttt{strtok}의 용례(\url{https://blockdmask.tistory.com/382})와 더불어
\texttt{fprintf}의 용례(\url{https://jhnyang.tistory.com/314})를 이해하면
수월한 문제다.
\hfill\break
\begin{mdframed}\textbf{6. }
  프로그램을 시작하면 먼저 텍스트 파일 \texttt{data.mat}을 읽는다. 이 파일의
  첫 줄에는 양의 정수 $N\leq 100$이 주어지고, 이어진 $N$ 줄에는 각 줄마다
  $N$개의 정수가 주어진다. 즉 하나의 $N\times N$ 정수 행렬이 주어진다.
  그런 다음 화면에 프롬프트를 출력하고 일련의 사용자 명령을 처리하는
  프로그램을 작성하라.
\end{mdframed}
\begin{lstlisting}[style=C]
#include <stdio.h>
#include <stdlib.h>
#include <string.h>

#define MAX_SIZE 100
#define BUF_SIZE 256

void show_mat(int matrix[MAX_SIZE][MAX_SIZE], int size);
void colmax_mat(int matrix[MAX_SIZE][MAX_SIZE], int size);
void colmin_mat(int matrix[MAX_SIZE][MAX_SIZE], int size);
void rowmax_mat(int matrix[MAX_SIZE][MAX_SIZE], int size);
void rowmin_mat(int matrix[MAX_SIZE][MAX_SIZE], int size);
void slice_mat(int matrix[MAX_SIZE][MAX_SIZE], int size, int x, int p, int y, int q);

int main(){
  const char* exit = "exit";
  const char* show = "show";
  const char* colmax = "colmax";
  const char* colmin = "colmin";
  const char* rowmax = "rowmax";
  const char* rowmin = "rowmin";
  const char* slice = "slice"; 
  char matrix[MAX_SIZE][MAX_SIZE][BUF_SIZE]; 
  char buf[BUF_SIZE];
  int n;
  int width[MAX_SIZE] = {0};
  int numMat[MAX_SIZE][MAX_SIZE];
  FILE *input;
  input = fopen("data.mat", "r");
  if(fgets(buf, sizeof(buf), input)!=NULL)
    sscanf(buf, "%d", &n);
  for(int row=0; row<n; row++){
    if(fgets(buf, sizeof(buf),input)==NULL)
      break;
    char *token = strtok(buf, "\t");
    int col = 0;
    while(token != NULL && col<n){
      strcpy(matrix[row][col], token);
      int tokLen = strlen(matrix[row][col]);
      if(tokLen > width[col])
        width[col] = tokLen;
      token = strtok(NULL, "\t");
      col++;
    }
    char* temp = strdup(matrix[row][0]);
    token = strtok(temp, " ");
    col = 0;
    while(token != NULL){
      int number = atoi(token);
      numMat[row][col]=number;
      token = strtok(NULL, " ");
      col++;
    }
  }
  while(1){
    char arg[10];
    printf("\n$ ");
    scanf("%s", arg);
    if(strcmp(arg, exit)==0)
      break;
    else if(strcmp(arg, show)==0)
      show_mat(numMat, n);
    else if(strcmp(arg, colmax)==0)
      colmax_mat(numMat, n);
    else if(strcmp(arg, colmin)==0)
      colmin_mat(numMat, n);
    else if(strcmp(arg, rowmax)==0)
      rowmax_mat(numMat, n);
    else if(strcmp(arg, rowmin)==0)
      rowmin_mat(numMat, n);
    else if(strcmp(arg, slice)==0){
      int x, p, y, q;
      scanf("%d %d %d %d", &x, &p, &y, &q);
      slice_mat(numMat, n, x, p, y, q);
    }
  }
  fclose(input);
}
void show_mat(int matrix[MAX_SIZE][MAX_SIZE], int size){
  for(int row=0; row<size; row++){
    for(int col=0; col<size; col++)
      printf("%d\t", matrix[row][col]);
    printf("\n");
  }
}
void colmax_mat(int matrix[MAX_SIZE][MAX_SIZE], int size){
  int max[MAX_SIZE];
  for(int i=0;i<size;i++)
    max[i] = matrix[0][i];
  for(int row=0;row<size;row++){
    for(int col=0;col<size;col++){
      if(max[col]< matrix[row][col])
        max[col] = matrix[row][col];
    }
  }
  for(int row=0;row<size;row++)
    printf("%d\t", max[row]);
  printf("\n");
}
void colmin_mat(int matrix[MAX_SIZE][MAX_SIZE], int size){
  int min[MAX_SIZE];
  for(int i=0;i<size;i++)
    min[i] = matrix[0][i];
  for(int row=0;row<size;row++){
    for(int col=0;col<size;col++){
      if(min[col]>matrix[row][col])
        min[col] = matrix[row][col];
    }
  }
  for(int row=0;row<size;row++)
    printf("%d\t", min[row]);
  printf("\n");
}
void rowmax_mat(int matrix[MAX_SIZE][MAX_SIZE], int size){
  int max[MAX_SIZE];
  for(int i=0;i<size;i++)
    max[i] = matrix[i][0];
  for(int row=0;row<size;row++){
    for(int col=0;col<size;col++){
      if(max[col]<matrix[col][row])
        max[col] = matrix[col][row];
    }
  }
  for(int row=0;row<size;row++)
    printf("%d\t", max[row]);
  printf("\n");
}
void rowmin_mat(int matrix[MAX_SIZE][MAX_SIZE], int size){
  int min[MAX_SIZE];
  for(int i=0;i<size;i++)
    min[i] = matrix[i][0];
  for(int row=0;row<size;row++){
    for(int col=0;col<size;col++){
      if(min[col]>matrix[col][row])
        min[col] = matrix[col][row];
    }
  }
  for(int row=0;row<size;row++)
    printf("%d\t", min[row]);
  printf("\n");
}
void slice_mat(int matrix[MAX_SIZE][MAX_SIZE], int size, int x, int p, int y, int q){
  for(int row=x; row<size; row=row+p){
    for(int col=y; col<size; col=col+q)
      printf("%d\t", matrix[row][col]);
  printf("\n");
  }
}
\end{lstlisting}
그냥 엄청 귀찮은 문제다.
\hfill\break
\begin{mdframed}\textbf{7. }
  강아지가 $N\times N$ 크기의 2차원 배열의 가운데 위치에서 출발한다.
  $N$은 홀수다. 상, 하, 좌, 우 4방향으로 인접한 셀들 중에서 방문한 적이 없는
  한 셀을 동일한 확률로 랜덤하게 선택하여 한 칸 이동한다. 배열의 가장자리
  셀에 도착하면 탈출에 성공한 것이다. 하지만 아무 곳으로도 이동할 수 없는
  상태에 처하면 탈출에 실패한 것이다. 입력으로 하나의 홀수 $N\leq100$을 받아서
  강아지가 탈출에 성공할 확률을 시뮬레이션으로 계산하는 프로그램을 작성하라. 
  실험 횟수는 10,000번으로 하라.
\end{mdframed}
\begin{lstlisting}[style=C]
#include <stdio.h>
#include <stdlib.h>
#include <time.h>
int escape(int N);
int main() {
  int N;
  scanf("%d", &N);
  int success = escape(N);
  double prob = (double)success/10000.0;
  printf("%.3lf\n", prob);
  return 0;
}
int escape(int N) {
  int count = 0;
  int x, y;
  srand(time(NULL));
  for (int i = 0; i < 10000; i++) {
    x = N/2;
    y = N/2;
    int visited[N][N];
    for(int row = 0; row < N; row++){
      for(int col = 0; col < N; col++)
        visited[row][col] = 0;
    }
    while (x > 0 && x < N-1 && y > 0 && y < N-1){
      visited[x][y] = 1;
      int dir[4][2] = {{-1, 0}, {1, 0}, {0, -1}, {0, 1}};
      int possDir[4][2];
      int possDirCnt = 0;
      for (int i=0; i<4; i++) {
        int nX = x + dir[i][0];
        int nY = y + dir[i][1];
          if (!visited[nX][nY]){
            possDir[possDirCnt][0] = nX;
            possDir[possDirCnt][1] = nY;
            possDirCnt++;
          }
        }
      if(possDirCnt == 0)
        break;
      int random = rand() % possDirCnt;
        x = possDir[random][0];
        y = possDir[random][1];
    }
    if(x == 0 || x == N - 1 || y == 0 || y == N-1)
      count++;
  }
  return count;
}
\end{lstlisting}
\texttt{time}을 통한 유사난수의 구현과 다중 조건 문장이 핵심이다.
\hfill\break
\begin{mdframed}\textbf{8. }
  입력 파일 \texttt{board.txt}에 오목판의 상태가 주어진다. 파일의 첫 줄에는
  바둑판의 크기 $N\leq19$가 주어지고, 이어진 $N$줄에는 각 줄마다 $N$개의
  정수 0, 1 혹은 2가 주어진다. 0은 빈자리를 표시하고 1은 검은 돌, 2는 흰 돌을
  표시한다. 주어진 상태가 검은 돌이 이긴 상태인지, 흰 돌이 이긴 상태인지, 혹은
  아직 아무도 못 이긴 상태인지 검사하여 Black, White, 혹은 Not Finished라고
  출력하는 프로그램을 작성하라. 둘 다 이긴 상태는 없다고 가정한다.
\end{mdframed}
\begin{lstlisting}[style=C]
#include <stdio.h>
#include <stdlib.h>
#define MAX_N 19
int check(int board[MAX_N][MAX_N], int N);
int main(){
  int N;
  FILE *inputFile;
  inputFile = fopen("board.txt", "r");
  fscanf(inputFile, "%d", &N);
  int board[MAX_N][MAX_N];
  for(int i=0; i<N; i++) {
    for(int j=0; j<N; j++)
      fscanf(inputFile, "%d", &board[i][j]);
  }
  int result = check(board, N);
  if(result == 1) 
    printf("Black\n");
  else if(result == 2)
    printf("White\n");
  else
    printf("Not Finished\n");
  fclose(inputFile);
  return 0;
}

int check(int board[MAX_N][MAX_N], int N){
  int i, j;
  for (i=0; i<N; i++){
    for (j=0; j<=N-5; j++){
      if (board[i][j] == 1 && board[i][j + 1] == 1 && board[i][j + 2] == 1 && board[i][j + 3] == 1 && board[i][j + 4] == 1)
        return 1;
      if (board[i][j] == 2 && board[i][j + 1] == 2 && board[i][j + 2] == 2 && board[i][j + 3] == 2 && board[i][j + 4] == 2) 
        return 2; 
    }
  }
  for (i=0; i<=N-5; i++){
    for (j=0; j<N; j++){
      if (board[i][j] == 1 && board[i + 1][j] == 1 && board[i + 2][j] == 1 && board[i + 3][j] == 1 && board[i + 4][j] == 1)
        return 1; 
      if (board[i][j] == 2 && board[i + 1][j] == 2 && board[i + 2][j] == 2 && board[i + 3][j] == 2 && board[i + 4][j] == 2)
        return 2; 
    }
  }
  for(i=0; i<=N-5; i++){
    for(j=0; j<=N-5; j++){
      if (board[i][j] == 1 && board[i + 1][j + 1] == 1 && board[i + 2][j + 2] == 1 && board[i + 3][j + 3] == 1 && board[i + 4][j + 4] == 1)
        return 1;
      if (board[i][j] == 2 && board[i + 1][j + 1] == 2 && board[i + 2][j + 2] == 2 && board[i + 3][j + 3] == 2 && board[i + 4][j + 4] == 2)
        return 2;
    }
  }
  for(i= 0; i<=N-5; i++){
    for (j=N-1; j>=4; j--){
      if (board[i][j] == 1 && board[i + 1][j - 1] == 1 && board[i + 2][j - 2] == 1 && board[i + 3][j - 3] == 1 && board[i + 4][j - 4] == 1)
        return 1; 
      if (board[i][j] == 2 && board[i + 1][j - 1] == 2 && board[i + 2][j - 2] == 2 && board[i + 3][j - 3] == 2 && board[i + 4][j - 4] == 2)
        return 2;
      }
  }
  return 0;
}
\end{lstlisting}
다중 조건문을 활용하면 되는 문제다.
\hfill\break
\begin{mdframed}\textbf{9. }
  \texttt{sample.html} 파일을 읽어서 파일에 등장하는 모든 HTML 태그들을 제거한
  후 \texttt{sample.txt}라는 이름의 파일로 저장하는 프로그램을 작성하라. 
  단 HTML 파일에서 태그를 제거하는 것 말고 다른 부분은 그대로 유지되어야 한다.
\end{mdframed}
\begin{lstlisting}[style=C]
#include <stdio.h>
#include <stdlib.h>
#include <stdbool.h>

int main() {
  FILE *input = fopen("sample.html", "r");
  FILE *output = fopen("sample.txt", "w");
  char c;
  bool in = false; 
  while((c = fgetc(input)) != EOF){
    if (c == '<')
      in = true;
    else if (c == '>') 
      in = false;
    else if (!in) 
      fputc(c, output);
  }
  fclose(input);
  fclose(output);
  return 0;
}
\end{lstlisting}
\hfill\break
\begin{mdframed}\textbf{10. }
  입력으로 텍스트파일 \texttt{input.txt}를 읽어서 왼쪽 정렬하여 
  \texttt{output.txt} 파일로 출력하는 프로그램을 작성하라. 출력 파일의 한 줄은
  80 문자를 초과해서는 안되며, 단어를 자르지 않는 한도 내에서 가능한 한
  최대한 80 문자에 가깝도록 맞춘다.
\end{mdframed}
\begin{lstlisting}[style=C]
#include <stdio.h>
#include <stdlib.h>
#include <string.h>
#define MAX 80
void align(FILE *input, FILE *output);

int main(){
  FILE *input = fopen("input.txt", "r");
  FILE *output = fopen("output.txt", "w");
  align(input, output);
  fclose(input);
  fclose(output);
  return 0;
}
void align(FILE *input, FILE *output){
  char line[MAX+1]; 
  int currLength = 0;
  while(fgets(line, sizeof(line), input) != NULL){
    int length = strlen(line);
    if(length > 0 && line[length-1] == '\n'){
      line[length-1] = '\0';
      length--;
    }
    int i = 0;
    while(i<length){
      int rem = MAX - currLength;
      if (rem >= length - i){
        fprintf(output, "%s", &line[i]);
        currLength += length-i;
        break;
      } else{
      fprintf(output, "%.*s\n", rem, &line[i]);
      currLength = 0;
      i += rem;
      }
    }
  }
}
\end{lstlisting}
\end{document}
