\documentclass[a4paper,chapter,atbegshi]{oblivoir}
\usepackage[dbl4x6]{fapapersize}
\usepackage{amsmath,amssymb,amsfonts,amsthm}
\usepackage{graphicx,xcolor,caption}
\usepackage{braket,hyperref,nicematrix}
\usepackage{euler,enumitem}
\setlist{nosep}
\hypersetup{
  colorlinks=true,linkcolor=blue,filecolor=magenta,urlcolor=cyan,
}
\newtheorem{defn}{정의}[chapter]
\newtheorem{theo}{정리}[chapter]
\newtheorem{thes}{논제}[chapter]
\newtheorem{axio}{공리}[chapter]

\title{\textsf{An Introduction to G\"odel's Theorems} 학습일지}
\author{김태원}
\date{최초 작성 : 2023년 8월 27일 \\ 최근 편집 : \today}

\begin{document}
\maketitle
\break
\tableofcontents
\chapter{계산}
\section{대각화}
함수 $f$가 정의역{\tiny domain} $\Delta$상의 원소를 공역{\tiny codomain} 
$\Gamma$상의 원소로 사상{\tiny maps to}한다는 말을 $f:\Delta\rightarrow\Gamma$와
같이 표기하고 $f$의 치역{\tiny range}은 $\{f(x)\in\Gamma|x\in\Delta\}$다.
$f$의 치역이 공역 $\Gamma$와 같다면 $f$는 전사{\tiny surjective}다.
그리고 $\Delta$상의 상이한 원소를 $\Gamma$상의 상이한 원소로 사상하는 $f$는
단사{\tiny injective}다. $f$가 전사이고 단사라면 전단사{\tiny bijcetive}다.
이제 시작하자.
\begin{defn}\label{defn:1-1}
성질{\tiny property} $P$의 \emph{특성함수\tiny characteristic function}
$c_P:\mathbb{N}\rightarrow\{0,1\}$는 $n$이 성질 $P$를 만족하면 
$c_p(n)=0$\footnote{통상적으로 참을 $1$로 표현하지만 괴델과 불완전성을 다루는
1장에 한해 참을 $0$으로 나타낸다.}인 함수다. 이때 성질 $P$는 
$\mathbb{N}$\footnote{물론 $\mathbb{N}$은 모든 자연수의 집합을 나타낸다.}을
두 집합으로 분할{\tiny partition}한다.
\end{defn}
\begin{defn}\label{defn:1-2}
집합 $\Sigma$가 \emph{열거가능\tiny enumerable} 혹은 \emph{가산}이라는
필요충분조건{\tiny iff}은 $\Sigma$가 공집합이거나 전사 함수 $f:\mathbb{N}
\rightarrow\Sigma$가 존재한다는 것이다. 
\end{defn}
위 두 정의와 아래 정리는 기초집합론에서 자주 다루는 내용이다.
\begin{theo}\label{theo:1-1}
  자연수 순서쌍 $\langle i,j\rangle$의 집합은 가산이다.
\end{theo}
\begin{proof}
  순서쌍을 그림 \ref{fig:1-1}과 같이 지그재그 꼴의 대각선으로 배열한다.
  그리고
  \[
  \begin{matrix}
     0\mapsto\langle0,0\rangle\quad
     &1\mapsto\langle0,1\rangle\quad
     &3\mapsto\langle0,2\rangle
     &6\mapsto\langle0,3\rangle\\
     &2\mapsto\langle1,0\rangle\quad
     &4\mapsto\langle1,1\rangle
     &7\mapsto\langle1,2\rangle\\
     &&5\mapsto\langle2,0\rangle
     &8\mapsto\langle 2,1\rangle\\
     &&&9\mapsto\langle3,0\rangle
  \end{matrix}
  \quad\cdots
  \]
  와 같이 전단사 $f:\mathbb{N}\rightarrow\mathbb{N}^2$를 정의할 수 있다.
\end{proof}
\begin{figure}[h]
\[
\begin{matrix}
\langle 0,0\rangle &\rightarrow &\langle 0,1\rangle & &\langle0,2\rangle & &\langle 0,3\rangle &\cdots \\
&\swarrow & &\swarrow & &\swarrow & &\swarrow \\
\langle 1,0\rangle & &\langle 1,1\rangle & &\langle 1,2\rangle &&\langle1,3\rangle&\cdots \\
&\swarrow &&\swarrow &&\swarrow && \\
\langle2,0\rangle&&\langle2,1\rangle&&\langle2,2\rangle&&\langle2,3\rangle&\cdots\\
&\swarrow&&\swarrow&&&&\\
\langle3,0\rangle&&\langle3,1\rangle&&\langle3,2\rangle&&\langle3,3\rangle&\cdots\\
\vdots&\swarrow&\vdots&&\vdots&&\vdots&
\end{matrix}
\]
\caption{\label{fig:1-1}대각 논법}
\end{figure}
아래는 이른바 칸토어 정리다.
\begin{theo}\label{theo:1-2}
  가산이 아닌 무한집합이 존재한다.
\end{theo}
\begin{proof}
  $\mathbb{N}$의 멱집합 $\mathcal{P}$가 아래처럼 존재한다.
  \[
    X\in\mathcal{P}\iff X\subseteq\mathbb{N}.
  \]
  함수 $f:\mathbb{N}\rightarrow\mathcal{P}$가 존재하여 $\mathcal{P}$가 
  가산이라고 가정한다. 이어서 $\mathbb{N}$의 부분집합 $D$를 아래처럼
  둔다.
  \[
    D=\{n\in\mathbb{N}:n\notin f(n)\}.
  \]
  $D\in\mathcal{P}$이고 $f$가 $\mathcal{P}$상의 원소에 대해 가산이기에 어떤
  $d\in\mathbb{N}$가 존재하여 $f(d)=D$를 만족한다. 그리하여 모든 $n\in 
  f(d)$에 대해 아래와 같다. 
  \[
    n\in f(d) \iff n \notin f(n).
  \]
  모순이다. 따라서 $f$와 같은 열거 함수는 존재할 수 없고
  멱집합 $\mathcal{P}$는 가산일 수 없다. 즉 비가산{\tiny indenumerable}이다. 
\end{proof}
여기서 $D$를 대각{\tiny diagonal}집합이라고 한다. 대각집합을 직접 사용하지
않더라도 대각화라는 발상만으로 다시 증명할 수도 있다.
\begin{proof}
  무한 이진문자열{\tiny binary strings}의 집합 $\mathbb{B}$가 존재한다.
  열거 함수 $f:\mathbb{N}\rightarrow\mathbb{B}$가 존재한다고 가정한다.
  \[
    \begin{matrix}
      0\rightarrow b_0:\underline{0}110001010011\ldots\\
      1\rightarrow b_1:1\underline{1}00101001101\ldots\\
      2\rightarrow b_2:11\underline{0}0101100001\ldots\\
      \vdots
    \end{matrix}
  \]
  대각선을 따라 $n\in\mathbb{N}$을 $n$번째 문자열 $b_n\in\mathbb{B}$의 $n+1$번째
  자릿수{\tiny digit}로 사상한다. 그리고 이제 각 $n$번째 자릿수에 대해
  $0$과 $1$을 뒤바꾼다. 이렇게 대각자릿수를 뒤집은 각 문자열 $d_i$는 $b_0$과
  $1$번째 자릿수에 대해 다르고, $b_1$과 $2$번째 자릿수에 대해 다르고, $b_2$와
  $3$번째 자릿수에 대해 다르다. 따라서 모든 $n\in\mathbb{N}$을 
  $b_n\in\mathbb{B}$에 대해 사상한 집합은 $d_i\in\mathbb{B}$를 포함하지 않는다.
  그리하여 $f:\mathbb{N}\rightarrow\mathbb{B}$는 (전사가 아니라서) 열거함수가
  아니다. 모순이다. 따라서 열거함수 $f:\mathbb{N}\rightarrow\mathbb{B}$는
  존재하지 않는다. 즉 $\mathbb{B}$는 비가산이다.
\end{proof}
\section{계산가능성}
대각화라는 발상은 온갖 알고리즘에서 유용하게 쓰인다. 우선 알고리즘을 
부분함수\footnote{부분함수{\tiny partial function}란 정의역상의 인자{\tiny
argument}에 대해 출력이 존재하지 않을 수도 있는 사상 $f$다.}에 
대한 계산{\tiny computation}이라고 대단히 추상적으로 정의할 수 있다. 또한
계산가능성{\tiny computability}을 컴퓨터의 크기나 속도와
완전히 무관하게 대단히 추상적으로 정의할 수 있다. 이처럼 급진적인 추상화는
알고리즘 및 계산의 가능성과 한계에 대해 내놓을 수 있는 모든 주장을 극단적으로
강화한다. 이런 맥락에서 컴퓨터 혹은 계산모델{\tiny computation model}로는
튜링장치{\tiny Turing Machine}를 전제한다.
\begin{thes}[튜링 1936]
  효과적으로 계산가능하다고 비형식적으로 말할 수 있는 수치함수는 실상
  전부 적절한 튜링장치로 계산가능한 함수다.
\end{thes}
유의해야 하는 표현은 \emph{수치함수\tiny numerical function}다. 어차피 모든
비수치{\tiny nonnumerical} 객체 $X$를 어떤 자연수(의 열)로 사상 혹은 부호화할
수 있기 때문이다. 또한 튜링장치에 의한 계산가능성과 동치인 개념이 그냥
깔끔하게 함수의 계산가능성이 아니라 `효과적인' 계산가능성이므로
거추장스러운 `효과적인'이라는 표현이 졸졸 따라다닐 예정이다. 
여기서 `효과'란 성능 따위가 아니라 말 그대로 어떤 효과를 지닌다는
뜻이다. 
\section{결정가능성}
효과적인 계산가능성 개념에 기초하여 효과적인 결정가능성 개념을 정의한다.
\begin{defn}\label{defn:1-3}
  성질이 \underline{효과적으로 결정가능하다\tiny effectively decidable}는 
  필요충분조건은 성질의 특성함수가 효과적으로 계산가능하다는 것이다.
\end{defn}
\begin{defn}\label{defn:1-4}
  집합 $\Sigma$가 \underline{효과적으로 결정가능하다}는 필요충분조건은
  $\Sigma$의 성질에 대한 특성함수 $c_{\Sigma}$가 효과적으로 계산가능하다는
  것이다.
\end{defn}
효과적인 결정가능성은 특성함수의 효과적인 계산가능성으로 정의되고 이 사실은
$\mathbb{N}$의 모든 유한 부분집합에 대해 번역될 수 있다. 
\begin{theo}\label{theo:1-3}
  모든 유한한 자연수 집합은 효과적으로 결정가능하다.
\end{theo}
\begin{proof}
  $\Sigma$가 유한한 자연수 집합이라고 하자.
  정의 \ref{defn:1-1}에 의해 특성함수 $c_{\Sigma}$는 $\Sigma\subset\mathbb{N}$에
  대해 항상 $1$이나 $0$의 값을 지닌다. $\Sigma$가 유한집합이기 때문에 특성함수
  $c_{\Sigma}$는 무차별대입{\tiny brute-force} 알고리즘으로 $1$이나 $0$으로
  효과적으로 계산가능\footnote{`계산가능성'과 `알고리즘의 존재성'이 대단히 비슷한
  쓰임새를 지닌다는 사실을 확인할 수 있다.}하다. 따라서 정의 \ref{defn:1-4}에
  의해 모든 유한한 자연수 집합은 효과적으로 결정가능하다.
\end{proof}
한 집합의 효과적인 결정가능성은 그 여집합{\tiny complement}의 결정가능성도
보장한다.
\begin{theo}\label{theo:1-4}
  $\Sigma$가 효과적으로 결정가능한 집합이라면 여집합
  $\overline{\Sigma}$도 효과적으로 결정가능한 집합이다.
\end{theo}
\begin{proof}
  효과적으로 결정가능한 집합 $\Sigma$가 있다고 하자. 정의 \ref{defn:1-4}에 의해
  $\Sigma$의 특성함수 $c_{\Sigma}$는 효과적으로 계산가능하다. $c_{\Sigma}$가
  효과적으로 계산가능하기에 다음처럼 정의된  $\overline{\Sigma}$의 
  특성함수 $\overline{c}$는 효과적으로 계산가능하다.
  \[
    \overline{c}(n) = 1-c_{\Sigma}(n)
  \]
  다시 정의 \ref{defn:1-4}에 의해 $\overline{\Sigma}$ 또한 효과적으로 결정가능한
  집합이다.
\end{proof}
\section{열거가능성}
이제 함수의 효과적인 계산가능성을 이용해 집합의 효과적인 열거가능성을 정의한다.
`계산가능 함수'가 실상 `알고리즘'이라는 사실을 적극적으로 쥐어짠다.
\begin{defn}\label{defn:1-5}
  집합 $\Sigma$가 \underline{효과적으로 열거가능하다\tiny effectively 
  enumerable}는 필요충분조건은 $\Sigma$가 공집합이거나 
  효과적인 계산가능 함수가 존재하여 $\Sigma$를 열거한다는 것이다.
\end{defn}
즉 정의 \ref{defn:1-2}를 돌이키면 $\Sigma$의 효과적인 열거가능성은
알고리즘으로서의 전사 함수 $f:\mathbb{N}\rightarrow\Sigma$의 존재성으로 정의된다.
\begin{theo}\label{theo:1-5}
  $\Sigma$가 효과적으로 결정가능한 집합이라면 효과적으로 열거가능하다.
\end{theo}
\begin{proof}
  효과적으로 결정가능한 집합 $\Sigma$가 있다고 하자. $s\in\Sigma$를
  고정하고 정의 \ref{defn:1-4}에 의해 임의의 입력 $n$에 대해
  $n$이 $\Sigma$에 속하는지 나타내는 특성함수 $c_{\Sigma}$를
  효과적으로 계산할 수 있는 알고리즘 $\Pi$를 구성한다. $\Pi$는
  아래처럼 정의된다.
  \[
    \Pi(n) = \begin{cases}n\quad n\in\Sigma\textrm{인 경우}\\
    s\quad n\notin\Sigma\textrm{인 경우}\end{cases}
  \]
  $\Pi$의 정의역은 $\mathbb{N}$이고 치역은 $\Sigma$다. 따라서
  정의 \ref{defn:1-2}와 \ref{defn:1-5}에 의해 
  $\Sigma$는 효과적으로 열거가능하다.
\end{proof}
알고리즘의 정의역이라는 개념이 증명에서 중요하게 쓰인다는 사실을
확인할 수 있다. 알고리즘을 한 번 더 사용해 보자. 
\begin{theo}
  $\Sigma$와 그 여집합 $\overline{\Sigma}$ 모두 효과적으로 열거가능한
  집합이면 $\Sigma$는 효과적으로 결정가능하다.
\end{theo}
\begin{proof}
  열거가능 집합 $\Sigma$이 존재하며 여집합 $\overline{\Sigma}$ 또한
  열거가능하다고 하자. 정의 \ref{defn:1-5}에 의해 $\Sigma$와 $\overline{\Sigma}$에
  대해 각각 계산가능함수 $f$와 $g$가 존재한다. $f$와 $g$가 출력할 수 있는
  있는 모든 $s$에 대해 아래와 같은 상황이 성립한다. 
  \[
    s\in\Sigma \iff s\notin\overline{\Sigma} 
  \]
  다시 말해 어떤 $m$이 존재하여 $f(m)=s$를 만족하거나 어떤 $n$이 존재하여
  $g(n)=s$를 만족한다. 따라서 정리 \ref{theo:1-4}에 의해 $c_{\Sigma}$ 혹은
  $c_{\overline{\Sigma}}$를 계산하는 알고리즘 $\Pi$를 $\Sigma$나
  $\overline{\Sigma}$에 대해 구성할 수 있다. 아래는 $\Sigma$에 대해 구성한
  $\Pi_{\Sigma}$다.
  \[
    \Pi_{\Sigma}(s)=
    \begin{cases}
      0\quad s\in\Sigma\textrm{인 경우} \\
      1\quad s\notin\Sigma\textrm{인 경우}
    \end{cases}
  \]
  그리고 임의의 공이 아닌 집합 $A$에 대해 $\overline{\overline{A}}=A$다.
  따라서 $\Sigma$는 정의 \ref{defn:1-4}에 의해 효과적으로 결정가능한 집합이다.
\end{proof}
이쯤 알고리즘의 정의역을 정의해 알고리즘을 더 분명하게 다룬다.
\begin{defn}
  알고리즘의 정의역은 입력 $n\in\mathbb{N}$에 대해 알고리즘이 언젠가는
  종결하며 어떤 수를 출력으로 내놓도록 하는 자연수의 집합이다. 
\end{defn}
알고리즘의 정의역은 실상 효과적으로 열거가능한 집합과 같은 말이다. 이 사실을
증명할 때는 알고리즘으로 알고리즘을 구성해 루프를 구성하는 기법이 유용하다.
또한 알고리즘의 정의상 종결을 유도해야 하기에 꽤 복잡해 보일 수 있다.
\begin{theo}\label{theo:1-7}
  $W$가 효과적으로 효과적으로 열거가능한 집합인 필요충분조건은 $W$가
  어떤 알고리즘의 정의역이라는 것이다.
\end{theo}
\begin{proof}
  필요충분조건을 양방향으로 증명한다.

  ($\Rightarrow$)
  $W$가 열거가능한 집합이라고 하자. 정의 \ref{defn:1-5}에 의해 아래가
  성립한다.
  \begin{enumerate}
    \item $W$가 공집합이거나
    \item 계산가능 함수 $f$가 존재하여 $W$를 열거한다. 즉 $i\in\mathbb{N}$에
      대해 아래와 같다.
      \[
        n\in W\iff f(i)=n
      \]
  \end{enumerate}
  첫 번째 경우 아무 출력도 내놓지 않는 알고리즘 아무거나 고르면 된다.
  두 번째 경우 함수 $f$를 계산하는 알고리즘 $\Pi$를 구성한다. 그리고 $\Pi$로
  다시 알고리즘 $\Pi^+$를 구성한다. 알고리즘 $\Pi^+$는 입력 $n\in W$에 대해
  $\Pi$로 루프{\tiny loop}하며 $f(0),f(1),f(2),\ldots$를 계산하다가 어떤 $i$에
  대해 $f(i)=n$인 경우 멈추고 $i$를 출력한다. 여기서 $\Pi^+$가 입력으로
  취하는 값 $n$은 $W$상의 임의의 원소다. 따라서 $W$는 $\Pi^+$의 정의역이다.

  ($\Leftarrow$) $W$가 알고리즘 $\Pi$의 정의역이라고 하자. 정의 \ref{defn:1-5}에
  의해 $W$가 공집합이라면 $W$는 열거가능하다. $W$가 공집합이 아니라고
  가정하고 $o\in W$를 고정하자.
  정리 \ref{theo:1-1}에 의해 $n\in\mathbb{N}$과 $\langle i,j\rangle$에 대해
  일대일대응이 존재한다.
  이 사실을 바탕으로 계산가능 함수 $\textrm{fst}(n)$과 $\textrm{snd}(n)$을 
  구성한다. 각각 $n$번째 쌍의 첫 번째 성분 $i$와 두 번째 성분 $j$를 반환하는
  함수다. 이들 함수로 새로운 알고리즘 $\Pi'$를 다음처럼 구성한다.
  \begin{quote}
  $\Pi'$는 우선 주어진 입력 $n\in\mathbb{N}$에 대해 $i=\textrm{fst}(n)$와
  $j=\textrm{snd}(n)$를 계산한다. 그리고 $i\notin W$라면 $o$를 출력한다. 
  $i\in W$라면 $i$를 입력으로 $\Pi$를 $j$번 루프한다. $\Pi$가 어떤 입력
  $i$에 대해 어떤 출력 $j$로 정지하면 $\Pi'$는 $i\in W$를 출력한다.
  \end{quote}
  즉 $\Pi'$는 $\mathbb{N}$를 $\Pi$의 정의역 $W$로 사상한다. 따라서 정의 
  \ref{defn:1-5}에 의해 $W$는 효과적으로 열거가능한 집합이다.
\end{proof}
아래는 당연한 사실이며 따로 증명하지 않겠다.
\begin{theo}\label{theo:1-8}
  모든 효과적으로 열거가능한 자연수 집합들의 집합 $\mathcal{W}$는 열거가능하다.
\end{theo}
그런데도 이 사실은 아래 따름정리를 유도하기에 중요하다.
\begin{theo}\label{theo:1-9}
  어떤 집합은 효과적으로 열거불가능하고 그래서 효과적으로 결정불가능하다.
\end{theo}
\begin{proof}
  정리 \ref{theo:1-2}에 의해 $\mathbb{N}$의 멱집합 $\mathcal{P}$는 열거가능하지
  않다. 즉 효과적으로 열거불가능한 집합이 존재한다. 그리고 정리 \ref{theo:1-5}의
  대우에 의해 $\mathcal{P}$는 효과적으로 결정불가능하다. 

  더 강력하게 증명할 수도 있다. 정리 \ref{theo:1-8}에 의해 모든 효과적으로
  열거가능한 자연수 집합들의 집합 $\mathcal{W}$는 열거가능하다. 그렇기에
  당연히 $\mathcal{W}\neq\mathcal{P}$이지만 더 중요한 함의는 바로 $\mathcal{W}
  \subset\mathcal{P}$다. 즉 $\mathcal{W}$의 원소가 아니라서 효과적으로 
  열거불가능한 집합들이 존재하며 그렇기에 효과적으로 결정불가능한 집합들이
  존재한다.
\end{proof}
아래 정리에는 대각화 구성 과 루프 등 지금까지 학습한 기법이 총동원된다. 
\begin{theo}[열거가능집합의 근본 정리]\label{theo:1-10}
  효과적으로 열거가능한 집합 $K$가 존재하여 여집합 $\overline{K}$는
  효과적으로 열거불가능하다.
\end{theo}
\begin{proof}
    효과적으로 열거가능한 집합 $K$가 존재하면 그 여집합 $\overline{K}$가
    효과적으로 열거불가능하다는 사실 (i)를 증명하고 이 사실에 기초해 효과적으로
    열거가능한 집합 $K$의 존재성 (ii)를 증명한다.

  (i) 
  효과적으로 열거가능한 집합 $K$는 정리 \ref{theo:1-8}에서 구성한 모든 효과적으로
  열거가능한 집합들의 집합 $\mathcal{W}$상 임의의 $e$번째 원소 $We$라는 집합이다.
  이 사실을 아래처럼 나타낼 수 있다. 
  \[
    K:=\{e|e\in W_e\}.
  \]
  여집합의 정의상 모든 $e$에 대해 아래가 성립한다.
  \[
    e\in\overline{K}\iff e\notin W_e.
  \]
  이 사실은 $e$ 뿐만 아니라 모든 $W_e$에 대해 성립한다. 즉 $\overline{K}$는
  정리 \ref{theo:1-9}에서 언급한 $\mathcal{P}/\mathcal{W}$상의 원소다. 
  다시 말해 $\overline{K}$는 효과적으로 열거불가능하다.

  (ii)
  $\overline{K}$가 효과적으로 열거불가능하므로 $\overline{K}$는 $\mathbb{N}$
  전체일 수 없다. 그리하여 $K$는 공이 아닌 집합이다. $o$를 $K$상의 어떤
  원소로 고정하고 알고리즘 $\Pi''$를 아래처럼 정의한다.
  \begin{quote}
    주어진 입력 $n\in\mathbb{N}$에 대해 $i=\textrm{fst}(n),j=\textrm{snd}(n)$을
    계산한다. 그리고 $i\notin K$라면 $o$를 출력한다. 
    $i\in K$라면 알고리즘 $\Pi_i$를 찾아 입력 $i$에 대해 $j$번 루프한다.
    $\Pi_i$가 입력 $i$에 대해 $j$를 출력하며 정지하면
    $\Pi''$는 $i$를 출력한다.
  \end{quote}
  즉 $\Pi''$는 $\mathbb{N}$을 각 $\Pi_i$의 치역인 $K$로 사상한다. 
  따라서 정의 \ref{defn:1-5}에 의해 $K$는 효과적으로 열거가능하다.
\end{proof}
이 증명을 통해 정리 \ref{theo:1-9}를 강화할 수 있다.
\begin{theo}
  어떤 효과적으로 열거가능한 집합은 결정가능하지 않다.
\end{theo}
\begin{proof}
  정리 \ref{theo:1-10}과 같이 효과적으로 열거불가능한 여집합을 지니는 임의의
  효과적으로 열거가능한 집합 $K$를 구성한다. 이에 $K$가 효과적으로 결정가능하다고
  가정하자. 그리고 $K$가 효과적으로 결정가능하면 정리 \ref{theo:1-4}에 의해
  그 여집합 $\overline{K}$ 또한 효과적으로 결정가능하다. 하지만 그렇다면
  정리 \ref{theo:1-5}에 의해 $\overline{K}$가 효과적으로 열거가능한 집합일 것이다.
  모순이다. 따라서 어떤 열거가능 집합은 결정불가능하다.
\end{proof}
이제 계산가능성, 결정가능성, 열거가능성이라는 강력한 도구로 이론과 언어의
문제를 다룬다.
\section{공리화된 이론}
\emph{효과적으로 공리화된 이론\tiny effectively axiomatized theory}은 
구문론{\tiny syntax} $\mathcal{L}$과 의미론{\tiny semantics} $\mathcal{I}$에 대해
결정가능성의 정의 \ref{defn:1-4}로 정의된다.
여기서 `wff'라고 나타내는 잘 형성된 공식{\tiny wel-formed formulae}은 
간신히 의미를 부여할 수 있거나 남아도는 자유변수 탓에 깔끔하지 않은 대상이다. 
중요한 것은 \emph{문장}이다. 문장은 닫힌 wffs로서 자유변수가 남아돌지 않는다.
\begin{defn}\label{defn:1-7}
  $T$는 아래를 만족하는 경우만
  \underline{효과적으로 공리화된 이론}이다.
  \begin{enumerate}
    \item $T$를 표현하는 형식화된 언어 $L=\langle\mathcal{L},\mathcal{I}
      \rangle$가 존재하여 $\mathcal{L}$의 $\mathcal{L}$-wff가 무엇인지 
      효과적으로 결정가능하다.
    \item 어떤 $\mathcal{L}$-wff가 $T$의 공리인지 효과적으로 결정가능하다. 
    \item $T$의 증명체계가 존재하여 $\mathcal{L}$-wff 배열이 증명설계규칙을
      따르는지 효과적으로 결정가능하다.
    \item $\mathcal{L}$-wff 배열이 $T$의 공리로 증명을 구성할 수 있는지
      효과적으로 결정가능하다.
  \end{enumerate}
\end{defn}
이제 중요한 단어들을 빠르게 정의한다. 
\begin{defn}\label{defn:1-8}
  이론 $T$의 공리에서 논리적 증명체계로 문장 $\varphi$가 유도될 떄
  $\varphi$를 $T$의 \underline{정리\tiny theorem}라고 부르고
  $T\vdash\varphi$라고 표기한다. 
\end{defn}
\begin{defn}\label{defn:1-9}
  이론 $T$가 \underline{타당하다\tiny sound}는 필요충분조건은 $T$의 모든 정리가
  참이라는 것이다.
\end{defn}
\begin{defn}\label{defn:1-10}
  이론 $T$가 \underline{효과적으로 결정가능하다}는 필요충분조건은 $T$의 정리로
  존재하는 성질이 효과적으로 결정가능한 성질이라는 것이다. 다시 말해
  $T$의 언어로 주어진 임의의 문장 $\varphi$에 대해 $T\vdash\varphi$인지 결정하는
  알고리즘이 존재한다는 것이다.
\end{defn}
\begin{defn}\label{defn:1-11}
  이론 $T$가 문장 $\varphi$를 \underline{결정한다}는 필요충분조건은 
  $T\vdash\varphi$이거나 $T\vdash\neg\varphi$라는 것이다. 이론 $T$가
  $\varphi$를 \underline{올바르게 결정한다}는 말은 $\varphi$가 참이라면
  $T\vdash\varphi$이고 $\varphi$가 거짓이라면 $T\vdash\neg\varphi$라는 뜻이다.
\end{defn}
\begin{defn}\label{defn:1-12}
  이론 $T$가 \underline{부정완전\tiny negation-complete}하다는 필요충분조건은
  $T$가 제 언어의 모든 문장 $\varphi$를 결정한다는 것이다. 다시 말해 모든 문장
  $\varphi$에 대해 $T\vdash\varphi$이거나 $T\vdash\neg\varphi$라는 것이다.
\end{defn}
\begin{defn}\label{defn:1-13}
  $T$가 \underline{모순적\tiny inconsistent}이라는 필요충분조건은 어떤 문장
  $\varphi$가 존재하여 $T\vdash\varphi$와 $T\vdash\neg\varphi$를 모두 지닌다는
  것이다.
\end{defn}
이어 효과적인 열거가능성의 개념 \ref{defn:1-5}로 아래처럼 정리한다. 여기서
알파벳이란 언어에 대해 주어지는 임의의 유한집합이다.
\begin{theo}\label{theo:1-12}
  $T$가 효과적으로 공리화된 이론일 때 이 집합들은 효과적으로 열거가능하다.
  \begin{enumerate}
    \item $T$의 wff들의 집합
    \item $T$의 문장들의 집합
    \item $T$상에서 구성가능한 증명들의 집합
    \item $T$의 정리들의 집합 
  \end{enumerate}
\end{theo}
\begin{proof}
  $T$가 효과적으로 공리화된 이론이라고 하자. 정의 \ref{defn:1-7}에 의해
  $T$는 유한 알파벳으로 구성된 형식 언어를 지닌다. 그리고 유한 알파벳상의
  모든 문자열은 대각화 알고리즘을 통해 효과적으로 열거될 수 있다. 길이
  $1$의 문자열, 길이 $2$의 문자열, $\cdots$ 길이 $n$의 문자열들에 대해 
  각각 알파벳 순서로 효과적으로 열거하는 알고리즘 $\Pi_1$이다. 또한 다시
  정의 \ref{defn:1-7}에 의해 wff를 효과적으로 결정할 수 있는 알고리즘
  $\Pi_2$가 존재한다. 따라서 $\Pi_1$이 가능한 모든 문자열을 대각화하는
  가운데 $\Pi_2$로 wff를 효과적으로 결정하여 이들만 효과적으로 열거하는
  통합적인 알고리즘 $\Pi_3$을 구성한다. 여기서 `wff'를 `문장'으로
  대체하면 $1,2$의 증명이 끝난다.

  $T$상에서 구성가능한 증명은 wff의 배열과 같고 증명들의 집합은 wff의
  배열들의 집합과 같다. 각 배열 길이 $i$와 배열상의 wff의 길이 $j$와
  알파벳 순서로 효과적으로 열거하는 대각화 알고리즘 $\Pi_4$를 구성한다. 
  그리고 wff의 배열은 무차별대입 알고리즘과 wff 결정 알고리즘 $\Pi_2$로
  구성할 수 있다. 그리하여 $T$상에서 구성가능한 증명들이 효과적으로 
  열거가능하므로 각 wff 배열에 대해 증명설계규칙을 따르는지 효과적으로
  결정가능한 알고리즘 $\Pi_5$를 구성한다. 따라서 정의 \ref{defn:1-7}에 의해
  $3,4$의 증명이 끝난다.
\end{proof}
이론의 무모순성{\tiny consistency}과 부정완전성은 결정가능성을 보장한다.
\begin{theo}\label{theo:1-13}
  임의의 무모순적이고 효과적으로 공리화된 부정완전 이론 $T$는 효과적으로
  결정가능하다.
\end{theo}
\begin{proof}
  $T$가 임의의 무모순적이고 효과적으로 공리화된 부정완전 이론이라고 하자.
  정리 \ref{theo:1-12}에 의해 $T$의 정리를 효과적으로 열거하는 알고리즘 $\Pi_1$이
  존재한다. $\Pi_1$에 따라 정리들을 효과적으로 열거하는 가운데 부정완전의 정의
  \ref{defn:1-12}에 의해 $\phi$가 나타나면 $\neg\phi$는 정리가
  아니라고 결정하고 $\neg\phi$가 나타면 $\phi$는 정리가 아니라고 결정하는
  알고리즘 $\Pi_2$를 구성한다. 또한 무모순의 정의 \ref{defn:1-13}의 정의에
  의해 위와 같은 $\Pi_2$는 정리와 정리가 아닌 문장을 전부 분할하는 알고리즘이다.
  다시 말해 $T$의 언어로 주어진 임의의 문장 $\phi$에 대해 $T\vdash\phi$를
  결정하는 알고리즘이 존재한다. 따라서 결정가능성의 정의 \ref{defn:1-10}에 의해
  $T$는 효과적으로 결정가능하다.
\end{proof}
\section{표현과 포착}
앞서 언급한 구문론과 의미론의 쌍 $\langle \mathcal{L},\mathcal{A}\rangle$의 
의미를 명시할 필요가 있다. 이 쌍은 어떤 언어 $L_A$을 구성한다. $\mathcal{L}_A$는
표준 1차논리 구문론을 지닌다. $\mathcal{L}_A$의 논리 어휘는 아래와 같은
것들이다.
\[
  \mathsf{
  \wedge, \vee, \rightarrow, \leftrightarrow, a, b, \ldots, e, u, \ldots, z,
  \forall, \exists, =
}
\]
$\mathcal{L}_A$의 비논리 어휘는 아래와 같다.
\begin{enumerate}
  \item 상수 $\mathsf{0}$
  \item 다음 수{\tiny successor} 함수 $\mathsf{S}$\footnote{형식언어는 \textsf{sans-serif}체로 나타낸다}
  \item 함수표현식 $+, \times$
\end{enumerate}
$\mathcal{L}_A$의 항{\tiny term}은 `$\mathsf{0}$'과 `$\mathsf{S}$',
`$+$', `$\times$'로 만들 수 있는 변수다. 그 외 어떤 것도 항일 수 없다. 

여기서 `$=$'는 $\mathcal{L}_A$의 유일한 술어{\tiny predicate}다. 따라서
원자{\tiny atomic} wff는 모두 임의의 항 `$\mathsf{a}$', `$\mathsf{b}$'에
대해 `$\mathsf{a = b}$' 꼴로 서술된다. 일반적으로 wff는 원자 wff에 양화사와 
연결사를 1차논리의 표준에 따라 사용한 것이다. 

의미론 $\mathcal{I}_A$를 해석{\tiny interpretation}이라고 볼 수 있다.
$\mathcal{I}_A$는 항에 대해 값을 배정하거나 모든 $\mathcal{L}_A$ 문장에 대해
고유한 해석을 효과적으로 배정한다. 그리하여 $L_A$가 효과적으로 형식화된 언어일
수 있다.

아래 정의에서 $\mathsf{\varphi(\overline{n})}$은 $\mathsf{\varphi(x)}$상의
`$\mathsf{x}$'에 대해 수치 $n$을 부여한 것이다.
\begin{defn}\label{defn:1-14}
  성질 $P$가 하나의 자유변수를 지닌 언어 $L$상의 wff $\varphi(\mathsf{x})$로
  \underline{표현된다{\tiny expressed}}는 필요충분조건은 모든 $n$에 대해 아래와
  같다는 것이다.
  \begin{itemize}
    \item $n$이 성질 $P$를 지니면 $\varphi(\mathsf{\overline{n}})$은 참이다.
    \item $n$이 성질 $P$를 지니지 않는다면 $\neg\varphi(\mathsf{\overline{n}})$은
      참이다.
  \end{itemize}
\end{defn}
수치함수 $f$는 형식 언어에 대해 아래처럼 정의된다.
\begin{defn}\label{defn:1-15}
  수치함수 $f$가 언어 $L$상의 wff $\varphi(\mathsf{x,y})$로 표현된다는
  필요충분조건은 임의의 $m,n$에 대해 아래와 같다는 것이다.
  \begin{itemize}
    \item $f(m)=n$이면 $\varphi(\mathsf{\overline{m},\overline{n}})$이 참이다.
    \item $f(m)\neq n$이면 $\neg\varphi(\mathsf{\overline{m},\overline{n}})$이
      참이다.
  \end{itemize}
\end{defn}
이를 간단하게 정리할 수 있다.
\begin{theo}\label{theo:1-14}
  $L$이 성질 $P$를 표현할 수 있는 필요충분조건은
  $L$이 $P$의 특성함수 $c_P$를 표현할 수 있다는 것이다.
\end{theo}
\begin{proof}
  아래 두 가지를 확인해야 한다. 
  \begin{enumerate}
    \item $\varphi(\mathsf{x})$가 $P$를 표현하면 $\mathsf{(\varphi(x)\wedge y=0)
        \vee (\neg\varphi(x)\wedge y=1)}$이 $c_p$를 표현한다.
      \item $\psi(x,y)$가 $c_P$를 표현하면 $\mathsf{\psi(x,0)}$이 $P$를 표현한다.
  \end{enumerate}
  우선 $\varphi(\mathsf{x})$가 $P$를 표현한다고 하자. 정의 \ref{defn:1-14}에 의해

  $c_P(m)=0$이라면 $\varphi(\mathsf{\overline{m}})$이 참이고 
  $(\varphi(\mathsf{\overline{m})\wedge0=0)\vee
  (\neg\varphi(\overline{m})\wedge0=1)}$이 참이다.
  
  $c_P(m)\neq0$이라면 $\neg\varphi(\mathsf{\overline{m}})$이 참이고
  $(\mathsf{\varphi(\overline{m})\wedge0=0)\vee
  (\neg\varphi(\overline{m})\wedge0=1})$은 거짓이다. 그리하여 아래 같은 부정이
  참이다.
  \begin{align*}
  &\mathsf{\neg((\varphi(\overline{m})\wedge0=0)\vee(\neg\varphi(\overline{m})\wedge0=1))}\\
    \equiv&\mathsf{\neg(\varphi(\overline{m})\wedge0=0)\wedge\neg(\neg\varphi(\overline{m})\wedge0=1)}
  \end{align*}
  따라서 정의 \ref{defn:1-15}에 의해 $L$은 $c_P$를 표현할 수 있다.
  
  $\psi(\mathsf{x,y})$가 $c_P$를 표현한다고 하자. 정의 \ref{defn:1-15}에 의해
  \begin{align*}
    c_P(m)=n\Rightarrow\psi(\mathsf{\overline{m},\overline{n}}) \\
    c_P(m)\neq n\Rightarrow \neg\psi(\mathsf{\overline{m},\overline{n}})
\end{align*}
$m,n$은 임의로 주어지기에 $n=0$으로 둔다. 이에 아래가 성립한다.
\begin{align*}
  c_P(m)=0\Rightarrow\psi(\mathsf{\overline{m},0})\\
  c_P(m)\neq0\mathsf{\Rightarrow\neg\psi(\overline{m},0})
\end{align*}
따라서 정의 \ref{defn:1-14}에 의해 $L$이 성질 $P$를 표현한다.
\end{proof}
표현보다 중요한 것이 증명 혹은 어떤 성질의 \emph{포착\tiny capture}이다.
\begin{defn}
  이론 $T$가 성질 $P$를 wff $\mathsf{\varphi(x)}$로 포착하는 필요충분조건은
  임의의 $n$에 대해 아래가 성립하는 것이다.
  \begin{itemize}
    \item $n$이 성질 $P$를 지닌다면 $T\vdash\varphi(\mathsf{\overline{n}})$이다.
    \item $n$이 성질 $P$를 지니지 않는다면 $T\vdash\neg\varphi(\mathsf{\overline{n}})$이다.
  \end{itemize}
\end{defn}
포착과 표현의 차이를 명시하는 편이 낫겠다. 성질 $P$가 주어진 이론상 표현가능하다는
말은 그 이론의 \emph{언어}의 풍부함에 의존하고 $P$가 이론에 의해 포착가능하다는
말은 그 이론의 \emph{공리와 증명체계}의 풍부함에 의존한다. 즉 
$\mathsf{\varphi(x)}$가 타당한 이론 $T$상에서 성질 $P$를 포착한다면 
$\mathsf{\varphi(x)}$는 $P$를 표현한다.
\end{document}
