\documentclass[a4paper,chapter,kosection,atbegshi]{oblivoir}
\usepackage[dbl4x6]{fapapersize}
\usepackage{amsmath, amsfonts,mdframed,hyperref,enumitem}
\usepackage[semibold,tt=false]{libertine}
\usepackage{libertinust1math}
\setlist{nosep}
\hypersetup{
    colorlinks=true,linkcolor=cyan,
    filecolor=magenta,urlcolor=cyan,
}

\renewcommand\chaptername{강}

\begin{document}
\title{양자정보학 개론\thanks{원문: \url{https://www.scottaaronson.com/qclec.pdf}}}
\author{
    스콧 애론슨\thanks{코리 오스트로브와 파울로 알브스의 큰 도움을 받았다.}, 
    2018년 가을\\
    번역: 김태원
}
\date{\today}
\newpage
\maketitle\thispagestyle{empty}\newpage

\tableofcontents

\chapter{강의 개요 및 확장 처치-튜링 논제}
\begin{itemize}[label=\(\blacktriangleright\)]
    \item 양자정보학이라는 분야는 천성이 간학문적이다. (물리학, 전산학, 수학, 공학, 철학)
    \item 그저 유용한 장치나 알고리즘의 발명에 관한 것만은 아니다.
        다만 양자역학의 작용에 대한 명료화에 관한 것이기도 하다.
    \begin{itemize}
        \item 양자역학으로 할 수 있느냐 없느냐는 물음을 던지기 위한 것이자
        \item 양자역학 자체의 본성에 대한 더 나은 이해를 독려하기 위한 것이다.
    \end{itemize}
    \item 애론슨 교수는 양자정보학 연구의 이론적인 극단에 헌신합니다.
    \item 이론가들은 실험가들이 만드는 것을 알리고, 그것은 또한 이론가들의 질문에 다시 영향을 미친다.
\end{itemize}\hfill\break

오늘은 물리적 세계에 관해 ``자명한" 진술들을 명시한다.
그런 다음 양자역학이 이들 진술 가운데 몇몇만 놔두고 나머지는 뒤엎어 버리는 광경을
목도할 텐데, 이들 진술 간의 차이란 종종 아주 미묘하다!
우선...
\end{document}
