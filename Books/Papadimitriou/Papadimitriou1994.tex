\documentclass[a4paper,chapter,atbegshi,itemph]{oblivoir}
\usepackage[dbl4x6]{fapapersize}
\usepackage{amsmath,amssymb,amsfonts}
\usepackage{enumitem}
\setlist{nosep}
\usepackage{graphicx,wrapfig,xcolor,caption}
\graphicspath{{./IMG/}}
\usepackage{hyperref}
\hypersetup{
  colorlinks=true,
  linkcolor=blue,
  filecolor=magenta,
  urlcolor=cyan,
}
\usepackage{tikz}
\usetikzlibrary{arrows.meta, decorations.markings}
\usepackage[symbol]{footmisc}

\headsep = 0.16in
\makepagestyle{mystyle} 
\setlength{\headwidth}{\dimexpr\textwidth+\marginparsep+\marginparwidth\relax}
\makerunningwidth{mystyle}{\headwidth}
% DON'T USE \textsf HERE, USE ITS LONGFORM \sffamily
\makeevenhead{mystyle}{\thepage}{}{\leftmark}
\makeoddhead{mystyle}{\rightmark}{}{\thepage}
\makeatletter
\makepsmarks{mystyle}{%
  \createmark{chaptername chapter}{left}{shownumber}{}{}
  \createmark{section}{right}{shownumber}{}{ }
  \clearmark{subsection}
  \clearmark{subsubsection}
}
\makeatother
% for pages where chapters begin
% don't overwrite plain
\makepagestyle{plainplain}
\makerunningwidth{plainplain}{\headwidth}
\makeevenfoot{plainplain}{}{}{}
\makeoddfoot{plainplain}{}{}{}
\aliaspagestyle{chapter}{plainplain}
\pagestyle{mystyle}

\title{계산복잡도\\Computational Complexity}
\author{크리스토스 H. 파파디미트리우\\
Chistos H. Papadimitiriou\\
\small 캘리포니아 대학교 샌디에이고\\
\small \copyright 1994 Addison-Wesley Publishing Company, Inc.}
\date{번역 김태원\\\today}

\begin{document}
\begin{titlingpage}
  \maketitle
\end{titlingpage}
\frontmatter
\tableofcontents
\mainmatter
\renewcommand{\partpageend}{}
\part{알고리즘}
\vspace{5\baselineskip}\noindent
\emph{알고리즘 책은 계산복잡도를 다루는 장으로 끝나기 마련이니, 알고리즘에 관한
기본 사실 몇 가지 돌이키며 본고를 시작하는 편이 적절하겠다. 이어지는 세 장에서
우리의 목표는 간단하지만 중요한 요점을 조금 지적하는 것이다. 계산{\footnotesize
computational} 문제란 해결되어야 하는 것일 뿐만 아니라, 탐구할 가치가 있는
객체이기도 하다는 점이다. 문제와 알고리즘은 수학적으로 형식화되고 분석될 수 있다.
차례로 이를테면 언어{\footnotesize languages}나 튜링장치{\footnotesize Turing
machines}가 그렇다. 그리고 정확한 형식주의는 그닥 중요하지 않다. 다항시간
계산가능성{\footnotesize Polynomial-time computability}은 계산 문제에 대해 원하는
성질로, 실용적인 해결가능성{\footnotesize practical solvability}의 직관적인
개념과 동질이다. 여러 상이한 계산 모델{\footnotesize models}은 효율성의 다항
손실{\footnotesize polynomial loss}로 또 다른 모델을 시뮬레이션할 수 
있다---비결정론{\footnotesize nondeterminism}이라는 예외, 즉 제 시뮬레이션에 
지수시간{\footnotesize exponential time}을 요구하는 것으로 보이는 예외를 제외하면
말이다. 그리고 알고리즘을 아예 지니지 않는 문제가 존재하는데, 아무리 비효율적인
것조차 지니지 않는다.
}
\cleardoublepage
\chapter{문제와 알고리즘}
\emph{알고리즘은 문제를 풀기 위한 자세한 스텝별{\footnotesize step-by-step}
방법론이다. 다만 문제{\footnotesize problem}가 뭔가? 우리는 이 장에서 
중요한 예시 세 개를 소개한다}
\section{그래프 도달가능성}
그래프 $G=(V,E)$는 노드{\footnotesize nodes} $V$와 선분{\footnotesize edges}
$E$, 즉 노드 쌍의 집합이다 (이를테면 그림 \ref{fig:1-1}을 보라. 우리의 그래프는
모두 유한하고 유향{\footnotesize directed}이겠다). 여러 계산 문제는 그래프를
다룬다. 그래프에 관해 가장 기본적인 문제는 이것이다. 그래프 $G$와 두 노드
$1,n\in V$가 주어질 때 $1$에서 $n$까지 경로{\footnotesize path}는 존재하는가?
우리는 이 문제를 \circemph{도달가능성}{\footnotesize 
REACHABILITY}\footnote[2]{복잡도 이론에서, 계산 문제는 단지 풀어야 하는 것일 뿐
아니라, 다만 그 자체로 흥미로운 수학적 객체이기도 하다. 문제가 수학적 객체로
다뤄질 때, 그 이름을 대문자로 표기하겠다. [역자: 대문자 표기를 
드러냄표(\texttt{circemph})로 대체한다.]}이라고 부른다. 가령, 그림 
\ref{fig:1-1}에는 분명 노드 $1$에서 $n=5$까지 경로, 즉 $(1,4,3,5)$가 존재한다. 
만약 이 대신 선분 $(4,3)$의 방향을 역전하면, 그런 경로는 존재하지 않는다. 

대다수의 흥미로운 문제와 마찬가지로, \circemph{도달가능성}은 가능한 
\emph{일례}{\footnotesize \emph{instances}}의 무한 집합을 지닌다. 각 일례는
수학적인 객체로, (우리의 경우, 그래프와 그 두 노드로) 곧 우리가 질문을
묻고 답을 기대하는 대상이다. 이때 질문이 속한 특정 종류가 문제를 특징짓는다.
\circemph{도달가능성}은 ``네'' 혹은 ``아니오'' 가운데 하나의 답안을 요구하는
질문이라는 점에 유의하라. 이런 문제는 \emph{결정문제\footnotesize decision 
problems}라고 부른다. 복잡도 이론에서 보통 우리는 온갖 상이한 답안을 요구하는 
문제보단 결정문제만 다루는 편이 편리하게 통합적이고 단순하다고 본다. 그러니
결정문제는 본고에서 중요한 역할을 맡겠다.

\newpage

\begin{figure}[h]\centering
\begin{tikzpicture}[
  decoration={markings, mark=at position 1 with {\arrow{Stealth[length=3mm]}}},
  dot/.style={circle,draw=black,fill=white,node contents={},label=#1},
  every edge/.style = {draw, postaction=decorate}
]
  \node (a) at (0, 4) [dot = $1$];
  \node (b) at (0, 0) [dot=below:$2$];
  \node (c) at (4, 0) [dot=below:$3$];
  \node (d) at (4, 4) [dot = $4$];
  \node (e) at (7.5, 2) [dot=right:$5$];
  \path (a) edge (d);
  \path (b) edge (a);
  \path (b) edge (c);
  \path (c) edge (e);
  \path (d) edge (c);
  \path (e) edge (d);
\end{tikzpicture}
\caption{\label{fig:1-1}그래프.}
\end{figure}

\newpage
gg
\end{document}
