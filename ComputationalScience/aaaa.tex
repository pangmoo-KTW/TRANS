\documentclass[14pt,twoside,reqno]{amsart}
\begin{document}
\section{lala}
\begin{itemize}
  \item GLM08b speculate \emph{passive} compnenents to enact the memory, such
    that after a signal is sent into the device, it \emph{propagates
    without external control to complete the memory access}.
  \item Results on QRAM are scattered.
  \item Classically we are used to ``gates" being static physical components
    that data propagates through, just as depicted in circuit diagrams.
    Applying that intuition to quantum computing is unjustified.
  \item In contrast, the memory peripheral framework of JS19 models quantum
    computer as data are static physical components, and gates are operations
    that a controller applies to the data.
  \item In this framework a quantum computer is a physical object (e.g., some
    Hilbert space) called a \emph{memory peripheral} that evolves independently
    under som Hamiltonian.
  \item There is also a memory controller that can choose to intervene on the
    object, either by applying a quantum channel from some defined set,
    or modifying the Hamilton.
  \item The crux of QRAM is that if a query state is a superposition over
    all addresses, then for the device or circuit to respond appropriately,
    it must perform a memory access over \emph{all} addresses simultaneously.
  \item imagining the memory laid out in space, a QRAM access must transfer
    some information to each of the bits in memory if it hopes to correctly
    perform a superposition of memory accesses. 
  \item This sets up an immediate contrast to classical memory, which can
    instead adaptively direct a signal through different parts of a memory
    circuit.
\end{itemize}
\end{document}
